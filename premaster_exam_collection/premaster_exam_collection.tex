\documentclass[pdftex,12pt,a4paper]{article}


%%%%%%%%%%%%%%%%%%%%%%%  Загрузка пакетов  %%%%%%%%%%%%%%%%%%%%%%%%%%%%%%%%%%

%\usepackage{showkeys} % показывать метки в готовом pdf 

\usepackage{etex} % расширение классического tex
% в частности позволяет подгружать гораздо больше пакетов, чем мы и займёмся далее

%\usepackage{mathtext} % русские буквы в формулах? (и без неё работает)
% Например, $x_{\text{один}}$

\usepackage{cmap} % для поиска русских слов в pdf
\usepackage{verbatim} % для многострочных комментариев
\usepackage{makeidx} % для создания предметных указателей
\usepackage[X2,T2A]{fontenc}
\usepackage[utf8]{inputenc} % задание utf8 кодировки исходного tex файла
\usepackage{setspace}
\usepackage{amsmath,amsfonts,amssymb,amsthm}
\usepackage{mathrsfs} % sudo yum install texlive-rsfs
\usepackage{dsfont} % sudo yum install texlive-doublestroke
\usepackage{array,multicol,multirow,bigstrut} % sudo yum install texlive-multirow
\usepackage{indentfirst} % установка отступа в первом абзаце главы
\usepackage[british,russian]{babel} % выбор языка для документа
\usepackage{bm}
\usepackage{bbm} % шрифт с двойными буквами
\usepackage[perpage]{footmisc}

\usepackage{dcolumn} % центрирование по разделителю для apsrtable

% создание гиперссылок в pdf
\usepackage[pdftex,unicode,colorlinks=true,urlcolor=blue,hyperindex,breaklinks]{hyperref} 

% свешиваем пунктуацию 
% теперь знаки пунктуации могут вылезать за правую границу текста, при этом текст выглядит ровнее
\usepackage{microtype}

\usepackage{textcomp}  % Чтобы в формулах можно было русские буквы писать через \text{}

% размер листа бумаги
\usepackage[paper=a4paper,top=13.5mm, bottom=13.5mm,left=16.5mm,right=13.5mm,includefoot]{geometry}

\usepackage{xcolor}

\usepackage[pdftex]{graphicx} % для вставки графики 

\usepackage{float,longtable}
\usepackage{soulutf8}

\usepackage{enumitem} % дополнительные плюшки для списков
%  например \begin{enumerate}[resume] позволяет продолжить нумерацию в новом списке

\usepackage{mathtools}
\usepackage{cancel,xspace} % sudo yum install texlive-cancel

\usepackage{minted} % display program code with syntax highlighting
% требует установки pygments и python 

\usepackage{numprint} % sudo yum install texlive-numprint
\npthousandsep{,}\npthousandthpartsep{}\npdecimalsign{.}

\usepackage{embedfile} % Чтобы код LaTeXа включился как приложение в PDF-файл

\usepackage{subfigure} % для создания нескольких рисунков внутри одного

\usepackage{tikz,pgfplots} % язык для рисования графики из latex'a
\usetikzlibrary{trees} % tikz-прибамбас для рисовки деревьев
\usepackage{tikz-qtree} % альтернативный tikz-прибамбас для рисовки деревьев
\usetikzlibrary{arrows} % tikz-прибамбас для рисовки стрелочек подлиннее

\usepackage{todonotes} % для вставки в документ заметок о том, что осталось сделать
% \todo{Здесь надо коэффициенты исправить}
% \missingfigure{Здесь будет Последний день Помпеи}
% \listoftodos --- печатает все поставленные \todo'шки


% более красивые таблицы
\usepackage{booktabs}
% заповеди из докупентации: 
% 1. Не используйте вертикальные линни
% 2. Не используйте двойные линии
% 3. Единицы измерения - в шапку таблицы
% 4. Не сокращайте .1 вместо 0.1
% 5. Повторяющееся значение повторяйте, а не говорите "то же"



%\usepackage{asymptote} % пакет для рисовки графики, должен идти после graphics
% но мы переходим на tikz :)

%\usepackage{sagetex} % для интеграции с Sage (вероятно тоже должен идти после graphics)

% metapost создает упрощенные eps файлы, которые можно напрямую включать в pdf 
% эта группа команд декларирует, что файлы будут этого упрощенного формата
% если metapost не используется, то этот блок не нужен
\usepackage{ifpdf} % для определения, запускается ли pdflatex или просто латех
\ifpdf
	\DeclareGraphicsRule{*}{mps}{*}{}
\fi
%%%%%%%%%%%%%%%%%%%%%%%%%%%%%%%%%%%%%%%%%%%%%%%%%%%%%%%%%%%%%%%%%%%%%%


%%%%%%%%%%%%%%%%%%%%%%%  Внедрение tex исходников в pdf файл  %%%%%%%%%%%%%%%%%%%%%%%%%%%%%%%%%%
\embedfile[desc={Main tex file}]{\jobname.tex} % Включение кода в выходной файл
\embedfile[desc={title_bor}]{/home/boris/science/tex_general/title_bor_utf8.tex}

%%%%%%%%%%%%%%%%%%%%%%%%%%%%%%%%%%%%%%%%%%%%%%%%%%%%%%%%%%%%%%%%%%%%%%



%%%%%%%%%%%%%%%%%%%%%%%  ПАРАМЕТРЫ  %%%%%%%%%%%%%%%%%%%%%%%%%%%%%%%%%%
\setstretch{1}                          % Межстрочный интервал
\flushbottom                            % Эта команда заставляет LaTeX чуть растягивать строки, чтобы получить идеально прямоугольную страницу
\righthyphenmin=2                       % Разрешение переноса двух и более символов
\pagestyle{plain}                       % Нумерация страниц снизу по центру.
\widowpenalty=300                     % Небольшое наказание за вдовствующую строку (одна строка абзаца на этой странице, остальное --- на следующей)
\clubpenalty=3000                     % Приличное наказание за сиротствующую строку (омерзительно висящая одинокая строка в начале страницы)
\setlength{\parindent}{1.5em}           % Красная строка.
%\captiondelim{. }
\setlength{\topsep}{0pt}
%%%%%%%%%%%%%%%%%%%%%%%%%%%%%%%%%%%%%%%%%%%%%%%%%%%%%%%%%%%%%%%%%%%%%%



%%%%%%%% Это окружение, которое выравнивает по центру без отступа, как у простого center
\newenvironment{center*}{%
  \setlength\topsep{0pt}
  \setlength\parskip{0pt}
  \begin{center}
}{%
  \end{center}
}
%%%%%%%%%%%%%%%%%%%%%%%%%%%%%%%%%%%%%%%%%%%%%%%%%%%%%%%%%%%%%%%%%%%%%%


%%%%%%%%%%%%%%%%%%%%%%%%%%% Правила переноса  слов
\hyphenation{ }
%%%%%%%%%%%%%%%%%%%%%%%%%%%%%%%%%%%%%%%%%%%%%%%%%%%%%%%%%%%%%%%%%%%%%%

\emergencystretch=2em


% DEFS
\def \mbf{\mathbf}
\def \msf{\mathsf}
\def \mbb{\mathbb}
\def \tbf{\textbf}
\def \tsf{\textsf}
\def \ttt{\texttt}
\def \tbb{\textbb}

\def \wh{\widehat}
\def \wt{\widetilde}
\def \ni{\noindent}
\def \ol{\overline}
\def \cd{\cdot}
\def \fr{\frac}
\def \bs{\backslash}
\def \lims{\limits}
\DeclareMathOperator{\dist}{dist}
\DeclareMathOperator{\VC}{VCdim}
\DeclareMathOperator{\card}{card}
\DeclareMathOperator{\sign}{sign}
\DeclareMathOperator{\sgn}{sign}
\DeclareMathOperator{\Tr}{\mbf{Tr}}
\DeclareMathOperator{\tr}{tr}


\def \xfs{(x_1,\ldots,x_{n-1})}
\DeclareMathOperator*{\argmin}{arg\,min}
\DeclareMathOperator*{\amn}{arg\,min}
\DeclareMathOperator*{\amx}{arg\,max}
\DeclareMathOperator{\trace}{tr}


\DeclareMathOperator{\Corr}{Corr}
\DeclareMathOperator{\sCorr}{sCorr}
\DeclareMathOperator{\sCov}{sCov}
\DeclareMathOperator{\sVar}{sVar}

\DeclareMathOperator{\argmax}{argmax}
\DeclareMathOperator{\Cov}{Cov}
\DeclareMathOperator{\Var}{Var}
\DeclareMathOperator{\corr}{Corr}
\DeclareMathOperator{\cov}{Cov}
\DeclareMathOperator{\var}{Var}
\DeclareMathOperator{\bin}{Bin}
\DeclareMathOperator{\Bin}{Bin}
\DeclareMathOperator{\rang}{rang}
\DeclareMathOperator*{\plim}{plim}
\DeclareMathOperator{\MSE}{MSE}

\providecommand{\iff}{\Leftrightarrow}
\providecommand{\hence}{\Rightarrow}

\def \ti{\tilde}
\def \wti{\widetilde}

\def \mA{\mathcal{A}}
\def \mB{\mathcal{B}}
\def \mC{\mathcal{C}}
\def \mE{\mathcal{E}}
\def \mF{\mathcal{F}}
\def \mH{\mathcal{H}}
\def \mL{\mathcal{L}}
\def \mN{\mathcal{N}}
\def \mU{\mathcal{U}}
\def \mV{\mathcal{V}}
\def \mW{\mathcal{W}}


\def \R{\mbb R}
\def \N{\mbb N}
\def \Z{\mbb Z}
\def \P{\mbb{P}}
\def \p{\mbb{P}}
\newcommand{\E}{\mathbb{E}}
\def \D{\msf{D}}
\def \I{\mbf{I}}

\def \QQ{\mbb Q}
\def \RR{\mbb R}
\def \NN{\mbb N}
\def \ZZ{\mbb Z}
\def \PP{\mbb P}


\def \a{\alpha}
\def \b{\beta}
\def \t{\tau}
\def \dt{\delta}
\newcommand{\e}{\varepsilon}
\def \ga{\gamma}
\def \kp{\varkappa}
\def \la{\lambda}
\def \sg{\sigma}
\def \sgm{\sigma}
\def \tt{\theta}
\def \ve{\varepsilon}
\def \Dt{\Delta}
\def \La{\Lambda}
\def \Sgm{\Sigma}
\def \Sg{\Sigma}
\def \Tt{\Theta}
\def \Om{\Omega}
\def \om{\omega}

%\newcommand{\p}{\partial}

\def \ni{\noindent}
\def \lq{\glqq}
\def \rq{\grqq}
\def \lbr{\linebreak}
\def \vsi{\vspace{0.1cm}}
\def \vsii{\vspace{0.2cm}}
\def \vsiii{\vspace{0.3cm}}
\def \vsiv{\vspace{0.4cm}}
\def \vsv{\vspace{0.5cm}}
\def \vsvi{\vspace{0.6cm}}
\def \vsvii{\vspace{0.7cm}}
\def \vsviii{\vspace{0.8cm}}
\def \vsix{\vspace{0.9cm}}
\def \VSI{\vspace{1cm}}
\def \VSII{\vspace{2cm}}
\def \VSIII{\vspace{3cm}}

\newcommand{\bls}[1]{\boldsymbol{#1}}
\newcommand{\bsA}{\boldsymbol{A}}
\newcommand{\bsH}{\boldsymbol{H}}
\newcommand{\bsI}{\boldsymbol{I}}
\newcommand{\bsP}{\boldsymbol{P}}
\newcommand{\bsR}{\boldsymbol{R}}
\newcommand{\bsS}{\boldsymbol{S}}
\newcommand{\bsX}{\boldsymbol{X}}
\newcommand{\bsY}{\boldsymbol{Y}}
\newcommand{\bsZ}{\boldsymbol{Z}}
\newcommand{\bse}{\boldsymbol{e}}
\newcommand{\bsq}{\boldsymbol{q}}
\newcommand{\bsy}{\boldsymbol{y}}
\newcommand{\bsbeta}{\boldsymbol{\beta}}
\newcommand{\fish}{\mathrm{F}}
\newcommand{\Fish}{\mathrm{F}}
\renewcommand{\phi}{\varphi}
\newcommand{\ind}{\mathds{1}}
\newcommand{\inds}[1]{\mathds{1}_{\{#1\}}}
\renewcommand{\to}{\rightarrow}
\newcommand{\sumin}{\sum\limits_{i=1}^n}
\newcommand{\ofbr}[1]{\bigl( \{ #1 \} \bigr)}     % Например, вероятность события. Большие круглые, нормальные фигурные скобки вокруг аргумента
\newcommand{\Ofbr}[1]{\Bigl( \bigl\{ #1 \bigr\} \Bigr)} % Например, вероятность события. Больше больших круглые, большие фигурные скобки вокруг аргумента
\newcommand{\oeq}{{}\textcircled{\raisebox{-0.4pt}{{}={}}}{}} % Равно в кружке
\newcommand{\og}{\textcircled{\raisebox{-0.4pt}{>}}}  % Знак больше в кружке

% вместо горизонтальной делаем косую черточку в нестрогих неравенствах
\renewcommand{\le}{\leqslant}
\renewcommand{\ge}{\geqslant}
\renewcommand{\leq}{\leqslant}
\renewcommand{\geq}{\geqslant}


\newcommand{\figb}[1]{\bigl\{ #1  \bigr\}} % большие фигурные скобки вокруг аргумента
\newcommand{\figB}[1]{\Bigl\{ #1  \Bigr\}} % Больше больших фигурные скобки вокруг аргумента
\newcommand{\parb}[1]{\bigl( #1  \bigr)}   % большие скобки вокруг аргумента
\newcommand{\parB}[1]{\Bigl( #1  \Bigr)}   % Больше больших круглые скобки вокруг аргумента
\newcommand{\parbb}[1]{\biggl( #1  \biggr)} % большие-большие круглые скобки вокруг аргумента
\newcommand{\br}[1]{\left( #1  \right)}    % круглые скобки, подгоняемые по размеру аргумента
\newcommand{\fbr}[1]{\left\{ #1  \right\}} % фигурные скобки, подгоняемые по размеру аргумента
\newcommand{\eqdef}{\mathrel{\stackrel{\rm def}=}} % знак равно по определению
\newcommand{\const}{\mathrm{const}}        % const прямым начертанием
\newcommand{\zdt}[1]{\textit{#1}}
\newcommand{\ENG}[1]{\foreignlanguage{british}{#1}}
\newcommand{\ENGs}{\selectlanguage{british}}
\newcommand{\RUSs}{\selectlanguage{russian}}
\newcommand{\iid}{\text{i.\hspace{1pt}i.\hspace{1pt}d.}}

\newdimen\theoremskip
\theoremskip=0pt
\newenvironment{note}{\par\vskip\theoremskip\textbf{Замечание.\xspace}}{\par\vskip\theoremskip}
\newenvironment{hint}{\par\vskip\theoremskip\textbf{Подсказка.\xspace}}{\par\vskip\theoremskip}
\newenvironment{ist}{\par\vskip\theoremskip Источник:\xspace}{\par\vskip\theoremskip}

\newcommand*{\tabvrulel}[1]{\multicolumn{1}{|c}{#1}}
\newcommand*{\tabvruler}[1]{\multicolumn{1}{c|}{#1}}

\newcommand{\II}{{\fontencoding{X2}\selectfont\CYRII}}   % I десятеричное (английская i неуместна)
\newcommand{\ii}{{\fontencoding{X2}\selectfont\cyrii}}   % i десятеричное
\newcommand{\EE}{{\fontencoding{X2}\selectfont\CYRYAT}}  % ЯТЬ
\newcommand{\ee}{{\fontencoding{X2}\selectfont\cyryat}}  % ять
\newcommand{\FF}{{\fontencoding{X2}\selectfont\CYROTLD}} % ФИТА
\newcommand{\ff}{{\fontencoding{X2}\selectfont\cyrotld}} % фита
\newcommand{\YY}{{\fontencoding{X2}\selectfont\CYRIZH}}  % ИЖИЦА
\newcommand{\yy}{{\fontencoding{X2}\selectfont\cyrizh}}  % ижица

%%%%%%%%%%%%%%%%%%%%% Определение разрядки разреженного текста и задание красивых многоточий
\sodef\so{}{.15em}{1em plus1em}{.3em plus.05em minus.05em}
\newcommand{\ldotst}{\so{...}}
\newcommand{\ldotsq}{\so{?\hbox{\hspace{-0.61ex}}..}}
\newcommand{\ldotse}{\so{!..}}
%%%%%%%%%%%%%%%%%%%%%%%%%%%%%%%%%%%%%%%%%%%%%%%%%%%%%%%%%%%%%%%%%%%%%%

%%%%%%%%%%%%%%%%%%%%%%%%%%%%% Команда для переноса символов бинарных операций
\def\hm#1{#1\nobreak\discretionary{}{\hbox{$#1$}}{}}
%%%%%%%%%%%%%%%%%%%%%%%%%%%%%%%%%%%%%%%%%%%%%%%%%%%%%%%%%%%%%%%%%%%%%%

\setlist[enumerate,1]{label=\arabic*., ref=\arabic*, partopsep=0pt plus 2pt, topsep=0pt plus 1.5pt,itemsep=0pt plus .5pt,parsep=0pt plus .5pt}
\setlist[itemize,1]{partopsep=0pt plus 2pt, topsep=0pt plus 1.5pt,itemsep=0pt plus .5pt,parsep=0pt plus .5pt}

% Эти парни затем, если вдруг не захочется управлять списками из-под уютненького enumitem
% или если будет жизненно важно, чтобы в списках были именно русские буквы.
%\setlength{\partopsep}{0pt plus 3pt}
%\setlength{\topsep}{0pt plus 2pt}
%\setlength{\itemsep}{0 plus 1pt}
%\setlength{\parsep}{0 plus 1pt}

%на всякий случай пока есть
%теоремы без нумерации и имени
%\newtheorem*{theor}{Теорема}

%"Определения","Замечания"
%и "Гипотезы" не нумеруются
%\newtheorem*{defin}{Определение}
%\newtheorem*{rem}{Замечание}
%\newtheorem*{conj}{Гипотеза}

%"Теоремы" и "Леммы" нумеруются
%по главам и согласованно м/у собой
%\newtheorem{theorem}{Теорема}
%\newtheorem{lemma}[theorem]{Лемма}

% Утверждения нумеруются по главам
% независимо от Лемм и Теорем
%\newtheorem{prop}{Утверждение}
%\newtheorem{cor}{Следствие} 

\title{Подборка вступительных экзаменов в магистратуру. \\Факультет экономики, НИУ-ВШЭ}
\author{Коллектив авторов}

\begin{document}
\parindent=0 pt % отступ равен 0

\maketitle

\tableofcontents




\section{2007}

\subsection{17.07.2007, вариант A}
\begin{enumerate}
\item Найдите 
\begin{equation}
\lim_{x \to 2}\frac{\sqrt[3]{6+x}-\sqrt{x+2}}{\sqrt[3]{25+x}-\sqrt{7+x}}
\end{equation}
\item Матрица вида А=
$\left( \begin{array}{cc}
a & b\\
c & -a
\end{array} \right)$ 
имеет собственное значение $\lambda_1=3$, которому соответствует собственный вектор 
$\left( \begin{array}{c}
1\\
1
\end{array} \right).$ Второй собственный вектор этой матрицы --- 
$\left( \begin{array}{c}
1\\
-2
\end{array} \right).$ 
Вычислите определитель данной матрицы.\\
\item Найдите стационарные точки функции $f(x,y)=e^{2x+3y}(8x^2-6xy+3y^2)$ и определите их тип.\\
\item Найдите минимумы и максимумы функции $f(x,y)=2x^2-3xy-2y^2$ при ограничении $x^2+y^2=10$.\\
\item Найдите решение дифференциального уравнения $xy'-6y=10x^4-16x^2$, удовлетворяющее условию $y(1)=0$. Постройте эскиз графика данного решения. \\
\item Найдите общее решение дифференциального уравнения $y''-4y'+3y=e^{2x}$.\\
\item Время, проводимое покупателем в супермаркете, можно считать нормально распределенной случайной величиной. Известно, что математическое ожидание этой случайной величины составляет 1 час 20 минут, а стандартное отклонение равно 15 минутам. Найдите при этих условиях вероятность того, что из трех незнакомых между собой покупателей хотя бы один проведет в супермаркете более полутора часов.\\
\item Функция плотности двумерной случайной величины $(X,Y)$ имеет вид\\
$p(x,y)=\begin{cases}
c(-x),\text{ если } x\in [0;1],y \in [1;2]\\
0,\text{ иначе }
\end{cases}$\\
Найдите:
\begin{enumerate}
\item значение константы с
\item вероятность того, что $Y\leq2X$
\item математическое ожидание $\E(Y)$
\end{enumerate}
\item Проверка 175 старых домов города показала, что в 56 из них электропровода требуют срочного ремонта. На уровне значимости 5\% проверьте гипотезу о том, что доля всех старых домов города, в которых требуется срочный ремонт электропроводки, составляет не менее 33\%.\\
\item По данным 22 наблюдений в рамках классической нормальной линейной регрессии была получена модель $\hat{Y}_i=\hat{\alpha}+\hat{\beta}X_i$ с коэффициентом детерминации $R^2=0.93$. Проверьте гипотезу об адекватности этой регресии
\end{enumerate}

\subsection{17.07.2007, вариант A1}
\begin{enumerate}
\item Найдите 
\begin{equation}
\lim_{x\to 0}\frac{\sqrt{1+\tg x}-\sqrt{1-\tg x}}{\sin x}
\end{equation}
\item Матрица вида А=
$\left( \begin{array}{cc}
0.6 & 0.8\\
c & d
\end{array} \right)$ 
коммутирует с матрицей B=
$\left( \begin{array}{cc}
0 & 1\\
-1 & 0
\end{array} \right)$, то есть выполняется равенство $A\cdot B=B\cdot A$. Найдите константы $c,d$ и матрицу $A^{-2}$ (матрицу, обратную матрице $A^2=A\cdot A$).\\

\item Найдите стационарные точки функции $f(x,y)=x^3+y^3+21x^2+18xy+21y^2$ и определите их тип.\\
\item Найдите минимумы и максимумы функции $f(x,y)=x^3\cdot y^4$ в области $x>0,y>0$ при ограничении $3x+4y=7$.\\
\item Найдите решение дифференциального уравнения $xy'+3y=x^2$, удовлетворяющее условию $y(1)=1/3$. Постройте эскиз графика найденного решения. \\
\item Найдите общее решение дифференциального уравнения $y''-y=e^{2x}$.\\
\item Синоптики Аляски и Чукотки независимо друг от друга предсказывают погоду (<<ясно>> или <<пасмурно>>) в Беринговом проливе, ошибаясь с вероятностями 0.1 и 0.2 соответственно. Их предсказания на завтра разошлись. Какова вероятность того, синоптики Аляски ошиблись?\\
\item Функция плотности двумерной случайной величины $(X,Y)$ имеет вид\\
$p(x,y)=\begin{cases}
c,\text{ при } x\in [0;1],y \in [0;1], y\geq x\\
0,\text{ в остальных случаях }
\end{cases}$\\
Найдите:
\begin{enumerate}
\item значение константы с
\item вероятность того, что $Y\leq 2X$
\item математическое ожидание $\E(Y)$
\end{enumerate}
\item Медицинское обследование 180 пациентов показало, что у 63 из них наблюдалось улучшение состояния после лечения новым препаратом. Найдите 95\% доверительный интервал для теоретической доли тех пациентов, у которых може наблюдаться такое улучшение.\\
\item По данным 30 наблюдений в рамках классической нормальной линейной регрессии была получена модель $\hat{Y}_i=\hat{\alpha}+\hat{\beta}X_i$, причём $\hat{\beta}=2.04, \hat{\sigma}_{\hat{\beta}}=0.75$. Проверьте адекватность этой регрессии.
\end{enumerate}


\subsection{17.07.2007, вариант В}
\begin{enumerate}
\item Найдите предел
\begin{equation}
\lim_{x\to 0} \frac{\sin 3x-\tg 4x}{\arcsin 2x}
\end{equation}
\item Найдите матрицу $X$ из уравнения $A\cdot X\cdot A'=B$, где $A=\left(\begin{array}{cc}
2 & 7\\
1 & 4
\end{array}\right), B=\left(\begin{array}{cc}
7 & 2\\
2 & 0
\end{array}\right), A'=\left(\begin{array}{cc}
2 & 1\\
7 & 4
\end{array}\right)$ --- матрица, транспонированная по отношению к матрице А.\\
\item Найдите минимумы и максимумы функции $f(x,y)=\ln x+\ln y-3x-y-6xy$.\\
\item Найдите решение дифференциального уравнения $y''+y'-2y=3$, удовлетворяющее условиям $y(0)=0$, $y'(0)=0$. Постройте эскиз графика найденного решения.\\
\item Время обслуживания одного вызова на междугородней телефонной станции можно считать нормально распределенной случайной величиной. При этом математическое ожидание составляет 1,5 минуты, а стандартное отклонение равно 0,5 минуты. Найдите при этих условиях вероятность того, что время обслуживания хотя бы одного из двух независимых вызовов составит более двух минут.\\
\item Функция плотности случайной величины $X$ имеет вид\\
$f(x,y)=\begin{cases}
c(1-|x|), |x|\leq 1\\
0, |x|>1.
\end{cases}$\\
Найдите значение константы $c$, математическое ожидание и дисперсию величины $X$.\\
\item Для случайной выборки, состоящей из 8 наблюдений, извлеченных из нормальной генеральной совокупности, был получен следующий 90\% доверительный интервал для математического ожидания $\mu$: $18,1<\mu< 18.9$. Постройте 95\% доверительный интервал для этого математического ожидания.\\
\item По данным 26 наблюдений в рамках классической нормальной регрессии была получена модель $\hat{Y}_i=\hat{\alpha}+\hat{\beta}X_i$, в которой $\hat{\beta}=2.0598$, $\hat{\sigma}_{\hat{\beta}}=0.0153$. На уровне значимости 5\% проверьте гипотезу
$\begin{aligned}
H_0&: \beta=2\\
H_a&: \beta\ne 2
\end{aligned}$
\end{enumerate}

\subsection{17.07.2007, вариант В1}
\begin{enumerate}
\item  Найдите предел
\begin{equation}
\lim_{x\to 0} \frac{\sqrt{1+x+x^2}-1}{x}
\end{equation}
\item Матрица $A=\left(\begin{array}{cc}
4 & a \\ 
-6 & -a
\end{array}\right)$ имеет собственное значение $\lambda=1$. Найдите константу $a$ и собственные значение матрицы $A^{-1}$.
\item Найдите стационарные точки в области $x>0$, $y>0$ и определите их тип для функции
\begin{equation}
f(x,y)=x^2y(4-x-y)
\end{equation}
\item Найдите решение дифференциального уравнения $y''+2y'+y=1$, удовлетворяющее условиям $y(0)=-1$, $y'(0)=4$. Постройте эскиз графика найденного решения.
\item Синоптики Аляски и Чукотки независимо друг от друга предсказывают погоду (<<ясно>> или <<пасмурно>>) в Беринговом проливе, ошибаясь с вероятностями 0.1 и 0.2 соответственно. Их предсказания на завтра совпали. Какова вероятность того, что эти предсказания ошибочны?
\item Функция плотности случайной величины $X$ имеет вид
\begin{equation}
f(x,y)=\left\{\begin{array}{c}
cx^2,\quad x\in[0;1] \\ 
0,\quad x\notin [0;1]
\end{array} \right.
\end{equation}
Найдите значение константы $c$, математическое ожидание и дисперсию величины $X$.
\item Для случайной выборки из 8 автомобилей средняя скорость на определенном участке трассы составила $\bar{X}=115$ км/ч, а выборочное стандартное отклонение --- $\hat{\sigma}=2$ км/ч. Предполагая нормальность закона распределения скорости постройте 95\% доверительный интервал для математического ожидания $\mu$ скорости.
\item По 28 наблюдениям в рамках классической нормальной линейной регрессии была получена модель $\hat{Y}_i=\hat{\alpha}+\hat{\beta}X_i$, в которой $\hat{\beta}=1.57$, $\hat{\sigma}_{\hat{\beta}}=0.05$. Вычислите коэффициент детерминации $R^2$.
\end{enumerate}

\subsection{17.07.2007, вариант В2}
\begin{enumerate}
\item  Найдите предел
\begin{equation}
\lim_{x\to 0} \frac{\sqrt{1+x}-\sqrt{1+x^2}}{\sqrt{1+x}-1}
\end{equation}
\item Матрица вида $A=\left(\begin{array}{cc}
2a & 1 \\ 
-3a & -1
\end{array}\right)$ имеет собственное значение $\lambda=2$. Найдите константу $a$ и собственные значение матрицы $A^{-1}$ (матрицы, обратной к матрице $A$).\\
\item Найдите стационарные точки функции $f(x,y)=x^2+xy-9x-3y$  и определите их тип.\\
\item Найдите решение дифференциального уравнения $y''-16y=32$, удовлетворяющее условиям $y(0)=0$, $y'(0)=0$. Постройте эскиз графика найденного решения.\\
\item Синоптики Аляски и Чукотки независимо друг от друга предсказывают погоду (<<ясно>> или <<пасмурно>>) в Беринговом проливе, ошибаясь с вероятностями 0.05 и 0.1 соответственно. Их предсказания на завтра совпали. Какова вероятность того, что эти предсказания верны?\\
\item Функция плотности случайной величины $X$ имеет вид
$p(x,y)=\begin{cases}
c/x^2,\text{ при } x\in [1;2]\\
0,\text{ в остальных случаях }
\end{cases}$
Найдите значение константы $c$, математическое ожидание и дисперсию случайной величины $X$.\\
\item Для случайной выборки из 10 студентов средний балл за контрольную работу составил $\bar{X}=71.2$ балла (по шкале в 100 баллов), причем выборочное стандартное отклонение $\hat{\sigma}=15.4$ балла. Постройте 95\% доверительный интервал для математического ожидания $\mu$ балла $X$ за эту контрольную работу (в предположении нормального закона распределения случайной величины $X$).\\ 
\item По 32 наблюдениям в рамках классической нормальной линейной регрессии была получена модель $\hat{Y}_i=\hat{\alpha}+\hat{\beta}X_i$, в которой $\hat{\beta}=0.32$, $\hat{\sigma}_{\hat{\beta}}=0.04$. Вычислите коэффициент детерминации $R^2$.
\end{enumerate}



\section{2008}

\subsection{22.07.2008, вариант А}
\begin{enumerate}
\item  Найдите предел 
\begin{equation}
\lim_{x\to 0} \left(\frac{1+6x+5x^2}{1-3x-x^2} \right)^{-3/x}
\end{equation}
\item В чемпионате по шахматам участвовало $n$ участников. Каждый участник сыграл с каждым один раз. Известно, что ничьих не было. По результатам матча судья составил матрицу $A$ размера $n\times n$ по принципу: $a_{ij}=1$, если игрок $i$ выиграл у игрока $j$; $a_{ij}=-1$, если игрок $i$ проиграл игроку $j$; диагональные элементы равны нулю, $a_{ii}=0$.
\begin{enumerate}
\item Найдите $\det(A)$ при $n=2$
\item Найдите $A+A^{t}$
\item Найдите $\det(A)$ при $n=1111$
\end{enumerate}
\item Найдите локальные минимумы и максимумы функции $f(x,y)=(x^2+y^2)e^{-4x^2-y^2}$
\item С помощью метода множителей Лагранжа найдите условные минимумы и максимумы функции $f(x,y,z)=5x^3y^5z^3$ при ограничении $x+y+5z=110$.
\item Решите дифференциальное уравнение $y'''+6y''-7y'=14+8e^{x}$
\item Для дифференциального уравнения $y'=(y+x)/(y-x)$ найдите
\begin{enumerate}
\item общее решение
\item частное решение, проходящее через точку $(x;y)=(0;1)$
\end{enumerate}
\item На острове Двупогодном погода бывает двух видов: пасмурная и ясная. Первого января губернатор острова «разгоняет»  тучи, поэтому в этот день на острове всегда ясно. В каждый последующий день погода меняется случайным образом согласно двум закономерностям. После пасмурного дня ясный наступает с  вероятностью 0.3, после ясного дня ясный наступает с вероятностью 0.8. 
\begin{enumerate}
\item Какова вероятность того, что второе января будет ясное?
\item Какова вероятность того, что второе января было ясное, если известно, что третье -- было пасмурное?
\end{enumerate}
\item Задана совместная функция плотности случайных величин $X$ и $Y$:
\begin{equation}
f_{X,Y}(x,y)=\left\{\begin{array}{c}
cx+y/2,\quad x,y\in [0;1] \\ 
0,\quad \text{иначе}
\end{array}  \right.
\end{equation}
Найдите $c$, $\E(XY)$ и $\P(XY<1/2)$
\item Исследуется зависимость спроса $Q$ на некоторый товар от его цены $P$. Предположим, что модель $\ln(Q)=\alpha+\beta\ln(P)+\varepsilon$ удовлетворяет всем условиям классической линейной регрессионной модели с нормально распределенной случайной ошибкой. Функция спроса оценивается по 10 наблюдениям. Известно, что 99\% доверительный интервал для коэффициента эластичности $\beta$ равен $(-1.44;-0.88)$.
\begin{enumerate}
\item Определите значение оценки $\hat{\beta}$ и оценки ее дисперсии.
\item Можно ли утверждать, что спрос зависит от цены товара?
\end{enumerate}
\item Распределение заработной платы работников подчиняется закону Эрланга с функцией плотности $f(x)=\frac{x}{\lambda^2}\exp(-x/\lambda)$ при $x>0$. Оцените значение параметра $\lambda$ по выборке $X_1$, $X_2$, \ldots, $X_n$ методом максимального правдоподобия. Будет ли полученная оценка несмещенной?

Примечание: $\int_0^{\infty} x^n \exp(-x/\lambda)=\lambda^{n+1}n!$
\end{enumerate}

\subsection{22.07.2008, вариант B}
\begin{enumerate}
\item  Найдите предел 
\begin{equation}
\lim_{x\to 0} \left(\frac{1+7x+2x^2}{1-x-x^2} \right)^{5/x}
\end{equation}
\item В чемпионате по шахматам участвовало $n$ участников. Каждый участник сыграл с каждым один раз. Известно, что ничьих не было. По результатам матча судья составил матрицу $A$ размера $n\times n$ по принципу: $a_{ij}=1$, если игрок $i$ выиграл у игрока $j$; $a_{ij}=-1$, если игрок $i$ проиграл игроку $j$; диагональные элементы равны нулю, $a_{ii}=0$.
\begin{enumerate}
\item Найдите $\det(A)$ при $n=2$
\item Найдите $A+A^{T}$
\item Найдите $\det(A)$ при $n=231$
\end{enumerate}
\item Найдите локальные минимумы и максимумы функции $f(x,y)=(x^2+y^2)e^{-x^2-9y^2}$
\item С помощью метода множителей Лагранжа найдите условные минимумы и максимумы функции $f(x,y,z)=6xy^{11}z^3$ при ограничении $x+2y+z=60$.
\item Решите линейное дифференциальное уравнение $y'''-3y''+2y'=4-e^{x}$
\item Для дифференциального уравнения $y'=(y+x)/(y-x)$ найдите
\begin{enumerate}
\item Найдите все решения дифференциального уравнения $y'=\frac{y+7x}{y-x}$
\item Выберите из них решение, проходящее через точку $(x;y)=(0;1)$
\end{enumerate}
\item На острове Двупогодном погода бывает двух видов: пасмурная и ясная. Первого января губернатор острова «разгоняет»  тучи, поэтому в этот день на острове всегда ясно. В каждый последующий день погода меняется случайным образом согласно двум закономерностям. После пасмурного дня ясный наступает с  вероятностью 0.4, после ясного дня ясный наступает с вероятностью 0.9. 
\begin{enumerate}
\item Какова вероятность того, что второе января будет ясное?
\item Какова вероятность того, что второе января было ясное, если известно, что третье -- было пасмурное?
\end{enumerate}
\item Задана совместная функция плотности случайных величин $X$ и $Y$:
\begin{equation}
f_{X,Y}(x,y)=\left\{\begin{array}{c}
cx+y/2,\quad x,y\in [0;1] \\ 
0,\quad \mbox{иначе}
\end{array}  \right.
\end{equation}
Найдите $c$, $\E(XY)$ и $\P(XY<1/2)$
\item Распределение доходов некоторой группы населения подчиняется закону Парето с $f(x)=\frac{1}{2\gamma}\left(\frac{2}{x}\right)^{1+1/\gamma}, x>2, 0<\gamma<1$.
Требуется оценить значение параметра $\gamma$ с помощью метода максимального правдоподобия по данным случайной выборки $n$ налоговых деклараций, заполненных респондентами из исследуемой доходной группы. Будет ли полученная оценка несмещенной?
\item Исследуется зависимость спроса $Q$ на некоторый товар от его цены $P$. Предположим, что модель $\ln(Q)=\alpha+\beta\ln(P)+\varepsilon$ удовлетворяет всем условиям классической линейной регрессионной модели с нормально распределенной случайной ошибкой. Функция спроса оценивается методом наименьших квадратов по 10 наблюдениям. Доверительный интервал для коэффициента эластичности $\beta$, соответствующий уровню доверия 95\%, принимает значение $(-2.4350,-1.8802)$. 
\begin{enumerate}
\item Определите значение МНК-оценки и оценки ее дисперсии для коэффициента эластичности.
\item На уровне значимости 1\% проверить гипотезу о единичной эластичности.
\end{enumerate}
\end{enumerate} 

\subsection{22.07.2008, вариант C}
\begin{enumerate}
\item  Найдите предел 
\begin{equation}
\lim_{x\to 0} \frac{e^{7x}+4e^{5x}-5e^{-4x}}{\arcsin \left(\arcsin(6x)\right)}
\end{equation}
\item Известно, что $A=
\left(\begin{array}{cc}
0.9 & 0.5\\
0.1 & 0.5
\end{array}\right)$.
\begin{enumerate}
\item Найдите собственные значения и собственные векторы матрицы $A$
\item Надйите $\lim_{n\to \infty} A^n$
\end{enumerate}
\item Найдите локальные минимумы и максимумы функции $f(x,y)=2x^5+5y^2+\frac{10}{xy}$\\
\item Найдите решение дифференциального уравнения $y'=\frac{6y-2xy^3-\sin(xy)-xy\cos(xy)}{3x^2y^2-6x+x^2\cos(xy)}$\\
\item Вася кидает дротик в мишень три раза. Известно, что во второй раз он попал ближе к центру, чем в первый раз. Какова условная вероятность того, что в третий раз он попадёт ближе к центру, чем в первый раз?\\
Указание: Предположить, что результаты бросков (расстояние от дротика до центра мишени) независимы друг от друга и имеют одинаковое непрерывное распределение.\\
\item Задана функция плотности случайной величины $X$:\\
$f_X(x)=\begin{cases}
c(x+x^2), x \in [0;1]
0, \text{иначе}
\end{cases}$.
\begin{enumerate}
\item Найдите значение константы $c$
\item Найдите  $\E(X^2)$
\item Найдите $\P(X>0.5)$
\end{enumerate}
\item Используя ежегодные данные об объеме импорта товаров $Y$ в личном располагаемом доходе $X$ в США за 1978 --- 1997 годы и предполагая,что модель $Y=\alpha+\beta X+\varepsilon$ удовлетворяет всем условиям классической линейной регрессионной модели с нормально распределенной ошибкой, исследователь получил методом наименьших квадратов следующее уравнени регрессии: $Y=-261.09+0.2452X$. Оценка дисперсии случайной ошибки $\varepsilon$ --- $\hat{\sigma}^2=475.48$, коэффициент детерминации $R^2=0.9388$.
\begin{enumerate}
\item На уровне значимости 5\% проверить гипотезу о независимости объема импорта от личного располагаемого дохода.
\item Предполагая, что в 1998 году располагаемый доход составил 2800 млрд. долларов, вычислить прогнозное значение для ожидаемого объема импорта. Какова точность полученного прогноза?
\end{enumerate}
\item В рекламе утверждалось, что из двух типов пластиковых карт: <<Visa>> и <<American Express>> богатые люди предпочитают второй, т.е. владельцы второго типа карт ежемесячно тратят больше денег. Выборочное обследование показало, что ежемесячные расходы по картам каждого типа достаточно хорошо описываются нормальным законом распределения. Средние месячные расходы 31 обладателя <<Visa>> оказались равны \$500 при выборочной дисперсии 39000 $\$^2$, а среднемесячные расходы 29 обладателей <<American Express>> --- \$ 580 при выборочной дисперсии 32000 $\$^2$. Проверить утверждение рекламы при 5\% уровне значимости.
\end{enumerate} 

\subsection{22.07.2008, вариант D}
\begin{enumerate}
\item  Найдите предел 
\begin{equation}
\lim_{x\to 0} \frac{e^{-7x}-3e^{-5x}+2e^{4x}}{\tg\tg(5x)}
\end{equation}
\item Известно, что $A=
\left(\begin{array}{cc}
0.3 & 0.5\\
0.7 & 0.5
\end{array}\right)$.
\begin{enumerate}
\item Найдите собственные значения и собственные векторы матрицы $A$
\item Надйите $\lim_{n\to \infty} A^n$
\end{enumerate}
\item Найдите локальные минимумы и максимумы функции $f(x,y)=2x^3+3y^2+\frac{6}{xy}$\\
\item Найдите решение дифференциального уравнения 
\begin{equation}
y'=\frac{18x^2y-y-2xy^2\cos (x^2y)}{x-6x^3+\sin(x^2y)+x^2y\cos(x^2 y)}
\end{equation} 
\item Вася кидает дротик в мишень три раза. Известно, что во второй раз он попал дальше от центра, чем в первый раз. Какова условная вероятность того, что в третий раз он попадёт ближе к центру, чем в первый раз?\\
Указание: Предположить, что результаты бросков (расстояние от дротика до центра мишени) независимы друг от друга и имеют одинаковое непрерывное распределение.\\
\item Задана функция плотности случайной величины $X$:\\
$f_X(x)=\begin{cases}
c(x+x^3), x \in [0;1] \\
0, \text{иначе}
\end{cases}$.
\begin{enumerate}
\item Найдите значение константы $c$
\item Найдите  $\E(X^2)$
\item Найдите $\P(X>0.5)$
\end{enumerate}
\item Используя ежегодные данные об объеме импорта товаров $Y$ в личном располагаемом доходе $X$ в США за 1978 --- 1997 годы и предполагая,что модель $Y=\alpha+\beta X+\varepsilon$ удовлетворяет всем условиям классической линейной регрессионной модели с нормально распределенной ошибкой, исследователь получил методом наименьших квадратов следующее уравнение регрессии: $Y=-261.09+0.2452X$ и оценку дисперсии $\widehat{\Var}(\hat{\beta})=0.0004$. 
\begin{enumerate}
\item Построить 95\% доверительный интервал для коэффициента наклона.
\item На уровне значимости 5\% проверить гипотезу о зависимости объема импорта от личного располагаемого дохода.
\end{enumerate}
\item Изучается эффективность нового метода обучения. У группы из 40 студентов, обучавшихся по новой методике, средний балл на экзамене составил 322.12, а выборочное стандартное отклонение --- 54.53. Аналогичные показатели для независимой выборки из 60 студентов того же курса, обучавшихся по старой методике, приняли значения 304.61 и 62.61 соответственно. Предполагая, что экзаменационный балл случайно выбранного студента хорошо описывается нормальным законом, проверить гипотезу об эффективности новой методики.
\end{enumerate} 


\section{2010}
\subsection{07.2010, вступительный экзамен}

! Здесь не хватает 9 и 10 задач

\begin{enumerate}
\item $\lim_{x\to 0}\frac{e^{x}\sin(x)-x(1+x)}{x^{3}}$

Ответ: $\frac{1}{3}$

\item Пусть $A=\left(\begin{array}{cc}9 & 1 \\ 1 & 9 \end{array}\right)$. Найдите $A^{-1}$, $A^{10}$, $A^{0.5}$, $A^{-0.5}$

Решение: $\lambda_{1}=8$, $\lambda_{2}=10$, $v_{1}=(1,-1)$, $v_{2}=(1,1)$

\item Найдите и классифицируйте экстремумы $f(x,y)=x^{3}+y^{3}+3xy$

Ответ: $(-1,-1)$ - локальный максимум

\item Найдите и классифицируйте экстремумы $f(x,y)=9x^{2}+9y^{2}+2xy$ при ограничении $x^{2}+y^{2}=1$.

Ответ: $(1/\sqrt{2},1/\sqrt{2})$ - максимум, $(-1/\sqrt{2},-1/\sqrt{2})$ - максимум, $(-1/\sqrt{2},1/\sqrt{2})$ - минимум, $(1/\sqrt{2},-1/\sqrt{2})$ - минимум

\item Решите уравнение $y'''-8y=0$

\item Решите систему: $x''=2y$ и $y''=-2x$.

Решение: $x(t)=e^{t}(C_{1}\sin(t)-C_{2}\cos(t))+e^{-t}(C_{4}\cos(t)-C_{3}\sin(t))$
$y(t)=e^{t}(C_{1}\cos(t)+C_{2}\sin(t))+e^{-t}(C_{3}\cos(t)+C_{4}\sin(t))$

\item Безработный индивид с вероятностью 20\% находит работу в течение ближайшего месяца (независимо от того, сколько времени он уже ищет работу). Индивид, имеющий работу, теряет ее в течение месяца с вероятностью 5\%. Известно, что на данный момент Петя является безработным.

Какова вероятность, что через два месяца Петя будет безработным?

Прошло два месяца, и Петя оказался безработным. Какова вероятность, что месяц назад он работал? (предполагается, что за месяц Петя может сделать только один переход между состояниями <<безработица>> и <<занятость>>).

Ответы: $0.65$, $1/65$

\item Контрольные камеры ДПС на МКАД зафиксировали скорость движения 6 автомобилей: 89, 83, 78, 96, 80, 78 км/ч. Предположим, что скорость распределена по нормальному закону.

Постройте 95\% доверительный интервал для средней скорости автомобилей, если истинная дисперсия равна 50 км/ч$^2$.

Постройте 80\% доверительный интервал для дисперсии скорости.

Ответ: $78.34<\mu<89.66$ и $27.94<\sigma^{2}<160.25$

\item ...
\item ...

\end{enumerate}

% Образец задания по высшей математике для программы <<Математическое моделирование>>
% 2010-тым годом датировал Кирилл Фурманов
\subsection{1 вариант,  июль 2010}
\begin{enumerate}
\item Вычислить предел $\lim_{x\to 0}\frac{e^x \sin x-x(1+x)}{x^3}$.\\
\item Пусть A =
$\left(\begin{array}{cc}
9 & 1\\
1 & 9
\end{array}\right)$. Найти:
\begin{enumerate}
\item $A^{-1}$
\item $A^{10}$
\item Такую симметрическую неотрицательно определенную матрицу $B$, что $A=B\cdot B$
\item Такую симметрическую неотрицательно определенную матрицу $C$, что $A^{-1}=C\cdot C$.
\end{enumerate}
\item Найти и классифицировать точки экстремума функции $f(x,y)=x^3+y^3+3xy$.\\
\item Найти и классифицировать экстремумы функции $f(x,y)=9x^2+9y^2+2xy$ при ограничении 
$x^2+y^2=1$.\\
\item Решить дифференциальное уравнение $y'''-8y=0$.\\
\item Решить систему дифференциальных уравнений 
$\left\{ \begin{aligned}
\ddot{x}&=2y\\
\ddot{y}&=-2x\\
\end{aligned} \right.$\\
\item Безработный индивид с вероятностью 20\% находит работу в течение ближайшего месяца (независимо от того, сколько времени он уже ищет работу). Индивид, имеющий работу, теряет её в течение месяца с вероятностью 5\%. Известно, что на данный момент индивид Петя является безработным.
\begin{enumerate}
\item Какова вероятность того, что через два месяца Петя тоже будет безработным?
\item По прошествии двух месяцев выясняется, что Петя является безработным. Какова вероятность того, что месяц назад он работал (предполагается, что за месяц Петя может сделать только один переход между состояниями <<безработица>> и <<занятость>>)?
\end{enumerate}
\item Контрольные камеры ДПС на МКАД зафиксировали скорость движения шести автомобилей: 89, 83, 78, 96, 80, 78 км/ч. Предполагается, что скорость распределена по нормальному закону.
\begin{enumerate}
\item Постройте 95\% доверительный интервал для средней скорости автомобилей, если известно, что настоящая дисперсия равна 50 $($км/ч$)^2$.
\item Постройте 80\% доверительный интервал для дисперсии скорости.
\end{enumerate} 
\item Имеется множество C, состоящее из $n$ элементов. Сколькими способами можно выбрать в C два подмножества A и B так, чтобы
\begin{enumerate}
\item множества A и B не пересекались
\item множество A содержалось бы в множестве B?
\end{enumerate}
\item В дереве по 2010 вершин степеней 3, 4 и 5 и нет вершин больших степеней. Сколько в этом дереве может быть
\begin{enumerate}
\item вершин степени 1?
\item вершин степени 2?
\end{enumerate}
Укажите все возможные варианты ответа.
\end{enumerate}

\subsection{2 вариант, июль 2010}
\begin{enumerate}
\item Вычислить предел $\lim_{x\to +\infty} x \sqrt{x}(\sqrt{x+1}+\sqrt{x-1}-2\sqrt{x})$.\\
\item Пусть $A=
\left(\begin{array}{ccc}
2 & 1 & 1\\
1 & 2 & 1\\
1 & 1 & 2
\end{array}\right)$ и $x=(x_1 x_2 x_3)^T$. Привести квадратичную форму $f(x)=x^T Ax$ к каноническому виду при помощи ортогонального преобразования (требуется указать, как сам канонический вид квадратичной формы, так и ортогональное преобразование, которое приводит форму к каноническому виду).\\
\item Найти и классифицировать точки экстремума функции $f(x,y)=3x^2-2x\sqrt{y}+y-8x+8$.\\
\item Найти и классифицировать экстремумы функции $f(x,y,z)=2x-y+9z^2$ при двух ограничениях $y+6xz=-1$ и $3z-2x=1$.\\
\item Решить дифференциальное уравнение $(x^2-y^2)dy + 2xydx=0$.\\
\item Решить систему дифференциальных уравнений 
$\left\{
\begin{aligned}
\ddot{x}+\dot{x}+\dot{y} & = 7\\
\dot{x}+\ddot{y}         & = e^t\\
\end{aligned}\right.$\\
\item Фирма производит микросхемы. Известно, что производство микросхем может находиться в одном из двух состояниях: нормальном (доля дефектных микросхем 10\%) и проблемном (доля дефектных микросхем 55\%). Для контроля состояния производства утром производится случайная выборка размером в 10 микросхем из продукции первого часа работы. Если из них 3 и более дефектные, производство останавливается до выяснения причины проблемы.
\begin{enumerate}
\item Найдите вероятность ложного срабатывания тревоги.
\item Найдите вероятность того, что проблемное состояние не будет идентифицировано.
\end{enumerate}
\item Доходность ценных бумаг на New York Фондовой бирже имеет нормальное распределение. В таблице приведены данные о доходности 10 видов ценных бумаг:\\ \\
\begin{tabular}{c|ccccccccccc}
  № & 1 & 2 & 3 & 4 & 5 & 6 & 7 & 8 & 9 & 10 & $\sum$ \\
  \hline
  X & 10 & 16 & 5 & 10 & 12 & 8 & 4 & 6 & 5 & 4 & 80 \\
  $X^2$ & 100 & 256 & 25 & 100 & 144 & 64 & 16 & 36 & 25 & 16 & 782\\
\end{tabular} \\

\begin{enumerate}
\item Найти точечные несмещенные и состоятельные оценки для математического ожидания и дисперсии доходности.
\item Найти 90\% доверительный интервал для математического ожидания доходности.
\end{enumerate}
\item Пусть $X_1,..,X_n$ --- выборка из нормально распределенной генеральной совокупности,т.е. $X_i~N(\mu,\sigma^2), i=1,...,n.$\\
Построены следующие оценки для математического ожидания $\mu$:\\
$\mu_1=\bar{X}, \mu_2=X_1, \mu_3=\frac{X_1}{2}+\frac{1}{2(n-1)}(X_2+...+X_n)$.
\begin{enumerate}
\item Какая из этих оценок является несмещенной?
\item Какая из этих оценок является наиболее эффективной?
\item Какая из этих оценок является состоятельной?
\end{enumerate}

\item Оценка зависимости выпуска фирмы от капитальных и трудовых затрат вида $Q=AK^{\beta_2}L^{\beta_3}$ с помощью модели $\ln Q=\beta_1+\beta_2\ln K+\beta_3\ln L+u$ по 40 наблюдениям дала следующие результаты (в скобках указаны стандартные ошибки коэффициентов регрессии):\\
$\ln Q=1.37+0.632 (0.257)\ln K+0.452(0.219)\ln L, R^2=0.98, \widehat{\Cov}(\hat{\beta}_2,\hat{\beta}_3)=-0.044$\\
\\На уровне значимости 5 \% проверить гипотезы
\begin{enumerate}
\item о значимости вклада труда/капитала в формирование выпуска
\item о наличии постоянной отдачи от масштаба.
\end{enumerate}

\end{enumerate}




\section{2011}

\subsection{22.07.2011, 1 вариант}
\begin{enumerate}
\item (10 баллов) Найдите и классифицируйте экстремумы функции $f(x,y,z)=2x-y+3z$ при ограничении $x^2+y^2+z^2=14$.\\
\item \begin{enumerate}
\item (4 балла) $n\times n$ матрица $A$ удовлетворяет соотношению $A^2-3A+I_n=0$. ($I_n$ --- единичная матрица). Может ли матрица $A$ быть вырожденной? Невырожденной?
\item (2 балла) $n\times n$ матрица $A$ удовлетворяет соотношению $A^2=0$. Следует ли отсюда, что $A=0? (n>1)$.
\item (4 балла) Множество многочленов степени 3, $M=\left\{f(x)=a_0+a_1x+a_2x^2+a_3x^3\right\}$ является линейным пространством относительно естественных операций сложения многочленов и умножения многочлена на число. Рассмотрим линейное преобразование $M\longrightarrow_A M$, такое, что $Af(x)=xf'(x)$. Найдите собственные числа и собственные векторы этого преобразования.
\end{enumerate}
\item Имеется матрица $A=\left(\begin{array}{cc}
4 & 2\\
2 & 4
\end{array}\right)$.
\begin{enumerate}
\item (3 балла) Найдите собственные числа и собственные векторы матрицы А.
\item (2 балла) Пусть $\vec{x}$ --- вектор-столбец подходящего размера, и $f(\vec{x})=\vec{x}^T A \vec{x}$. Какие значения может принимать функция $f(\vec{x})$ при произвольном векторе $\vec{x}$?
\item  (5 баллов) Обозначим через $||A||=[tr(A^TA)]^{1/2}$ норму матрицы $A$. ($A^T$ --- транспонированная матрица, $tr(B)$ --- след матрицы $B$.\\
Найдите $\lim_{n\to \infty}\frac{1}{||A^n||} A^n$.
\end{enumerate}
\item (10 баллов) Функция $y(x)$ на отрезке [0,2] удовлетворяет дифференциальному уравнению $y''+4y=0$, с граничными условиями: $y(0)=0$, $y'(0)=2$. Найдите $y(2)$.\\
\item (10 баллов) Функция $y(t)$ удовлетворяет уравнению $y''(t)-8y'(t)-9y(t)+7=0$, а функция $x(t)$ равна $x(t)=\frac{1}{4}(y'(t)-y(t)-1)$. Найдите функцию  $y(t)$, такую, что $x(0)=x'(0)=0$.\\
\item Два стрелка стреляют по мишени (каждый делает один выстрел). Для первого стрелка вероятность промаха составляет 0.3, для второго --- 0.5. Результаты выстрелов независимы.
\begin{enumerate}
\item (5 баллов) Какова вероятность того, что мишень будет поражена хотя бы одним из них?
\item (5 баллов) При выстреле двух стрелков мишень была поражена (хотя бы одним выстрелом). Какова вероятность того, что второй стрелок промахнулся?
\end{enumerate}
\item Случайная величина X принимает значения в интервале [0,2], и на этом интервале ее функция распределения равна $F(x)=cx^3$, где c --- некоторая константа.
\begin{enumerate}
\item (2 балла) Найдите  $\P(X<0.3\mid X<0.6)$.
\item (3 балла) Найдите $\Cov\left(X+1,\frac{1}{X}\right)$.
\end{enumerate}
\item Имеется случайная выборка $X_1,...,X_n$, где все  $X_i$ независимы и принимают значения 1, 3 и 5 со следующими вероятностями:\\
\begin{tabular}{cccc}
\hline
$x$ & 1 & 3 & 5\\
$\P(X_i=x)$ & $a$ & 0.2 & 0.8-$a$\\
\hline
\end{tabular}
 \begin{enumerate}
 \item (5 баллов) Какие значения являются допустимыми для параметра $a$? Постройте оценку параметра $a$ методом моментов. Обязательно ли оценка принадлежит области допустимых значений параметра $a$?
 \item (5 баллов) При каком значении $m$ оценка $\hat{a}=mX_1-1.15+\frac{1}{n-1}\sum_{i=2}^n X_i$ параметра $a$ является несмещённой?
 \end{enumerate}
\item Страховая компания выплачивает агентам комиссию. План возмещения убытков предполагает, что средние выплаты комиссий составят 32 тысячи долларов в год. Если средние выплаты будут меньше запланированных, то план потребуется изменить. Для проверки гипотезы о том, что средние выплаты равны 32 тысячам долларов, против альтернативной гипотезы о том, что средние выплаты меньше 32 тысяч, была сформирована случайная выборка из 49 агентов. В этой выборке средние выплаты комиссий составили 29.5 тысяч долларов, а несмещённая оценка дисперсии оказалась равна 36. Для проверки гипотезы выбран уровень значимости 5\%.
\begin{enumerate}
\item (3 балла) Рассчитайте статистику, с помощью которой проверяется указанная гипотеза.
\item (2 балла) Рассчитайте критическое значение этой статистики.
\item (2 балла) Выясните, даёт ли выборочное исследование основание для пересмотра плана пересмотра убытков.
\item (3 балла) Определите, при каких уровнях значимости основная гипотеза будет отвергаться, а при каких --- нет.
\end{enumerate}
\item При 20 наблюдениях с помощью МНК оценивается регрессионное уравнение $y_i=\beta_1+\beta_2 x_i+\beta_3 z_i+\beta_4 t_i+\epsilon_i$ при условиях на ошибки, соответствующих стандартной модели множественной регрессии. Полученные вектор оценок коэффициентов и оценка его матрицы ковариаций равны:\\
$\hat{\beta}=\left[ \begin{array}{c}
8.4739\\
20.8209\\
1.2309\\
-17.4765
\end{array}\right],
\widehat{\Var}(\hat{\beta})=\left[\begin{array}{cccc}
54.94838 & -24.62334 & -30.31618 & -0.628223\\
-24.62334 & 85.97937 & 8.523841 & -72.60611\\
-30.31618 & 8.523841 & 19.45426 & 6.176577\\
-0.628223 & -72.60611 & 6.176577 & 77.56094
\end{array}\right]$\\, а оценка дисперсии ошибок регрессии и коэффициент детерминации равны $s^2=117.0376, R^2=0.243649$.\\ 
Оценивание на тех же данных уравнения $y_i=\gamma_1+\gamma_2 t_i+\epsilon_i$ дало значение коэффициента детерминации $R^2=0.003539$.
\begin{enumerate}
\item (5 баллов) На 5\% уровне значимости тестируйте гипотезу $H_0:\beta_2=0$ против альтернативы $H_a: \beta_2 \ne 0$, а также тестируйте гипотезу $H_0:\beta_3=0$ против альтернативы $H_a: \beta_3 \ne 0$
\item (5 баллов) На 5\%-ном уровне значимости тестируйте гипотезу $H_0:\beta_2=\beta_3=0$ против альтернативы $H_a$: <<не $H_0$>>.
\item (5 баллов) На 5\%-ном уровне значимости тестируйте гипотезу $H_0:\beta_2=\beta_3$ против альтернативы $H_a$: $\beta_2>\beta_3$.
\end{enumerate}

\end{enumerate}

\subsection{22.07.2011, 2 вариант}
\begin{enumerate}
\item (10 баллов) Найдите и классифицируйте экстремумы функции $f(x,y,z)=x+2y+3z$ при ограничении $x^2+y^2+z^2=14$.\\
\item \begin{enumerate}
\item (4 балла) $n\times n$ матрица $A$ удовлетворяет соотношению $A^2+2A+I_n=0$. ($I_n$ --- единичная матрица). Может ли матрица $A$ быть вырожденной? Невырожденной?
\item (2 балла) $n\times n$ матрица $A$ удовлетворяет соотношению $A^2=A$. Следует ли отсюда, что  есть только две возможности: $A=0$ или $A=I_n? (n>1)$.
\item (4 балла) Множество многочленов степени 3, $M=\left\{f(x)=a_0+a_1x+a_2x^2+a_3x^3\right\}$ является линейным пространством относительно естественных операций сложения многочленов и умножения многочлена на число. Рассмотрим линейное преобразование $M\longrightarrow_A M$, такое, что $(Af)(x)=\frac{f(x)-f(0)}{x}$. Найдите собственные числа и собственные векторы этого преобразования.
\end{enumerate}
\item Имеется матрица $A=\left(\begin{array}{cc}
7 & 4\\
4 & 1
\end{array}\right)$.
\begin{enumerate}
\item (3 балла) Найдите собственные числа и собственные векторы матрицы А.
\item (2 балла) Пусть $\vec{x}$ --- вектор-столбец подходящего размера, и $f(\vec{x}=\vec{x}^T A \vec{x}$. Какие значения может принимать функция $f(\vec{x})$ при произвольном векторе $\vec{x}$?
\item  (5 баллов) Обозначим через $||A||=[tr(A^TA)]^{1/2}$ норму матрицы $A$. ($A^T$ --- транспонированная матрица, $tr(B)$ --- след матрицы $B$.\\
Найдите $\lim_{n\to \infty}\frac{1}{||A^n||} A^n$.
\end{enumerate}
\item (10 баллов) Функция $y(x)$ на отрезке [0,3] удовлетворяет дифференциальному уравнению $y''+3y'=0$, с граничными условиями: $y(0)=0, y'(3)=-3$. Найдите $y(3)$.\\
\item (10 баллов) Функция $y(t)$удовлетворяет уравнению $y''(t)-8y'(t)+12y(t)+8=0$, а функция $x(t)$ равна $x(t)=\frac{1}{2}(y'(t)-4y(t)-2)$. Найдите функцию  $y(t)$, такую, что $x(0)=x'(0)=0$.\\
\item На учениях два самолёта атакуют цель (каждый выпускает одну ракету). Известно, что первый самолёт поражает цель с вероятностью 0.6, а второй --- с вероятностью 0.4. Пусть самолёты поражают цель независимо друг от друга.
\begin{enumerate}
\item (5 баллов) Какова вероятность того, что цель будет поражена хотя бы одним самолётом?
\item (5 баллов) При разборе учений выяснилось, что цель была поражена только одним самолётом. Какова вероятность того, что цель поразил первый самолёт?
\end{enumerate}
\item Случайная величина X принимает значения в интервале [0,3], и на этом интервале ее функция распределения равна $F(x)=cx^2$, где c --- некоторая константа.
\begin{enumerate}
\item (2 балла) Найдите  $\P(X>2\mid X>1)$.
\item (3 балла) Найдите $\Cov\left(X^2+3,\frac{1}{X}\right)$.
\end{enumerate}
\item Имеется случайная выборка $X_1,...,X_n$, где все  $X_i$ независимы и принимают значения -1, 1 и 4 со следующими вероятностями:\\
\begin{tabular}{cccc}
\hline
$x$ & -1 & 1 & 4\\
$\P(X_i=x)$ & 0.3 & $a$ & 0.7-$a$\\
\hline
\end{tabular}
 \begin{enumerate}
 \item (5 баллов) Какие значения являются допустимыми для параметра $a$? Постройте оценку параметра $a$ методом моментов. Обязательно ли оценка принадлежит области допустимых значений параметра $a$?
 \item (5 баллов) При каком значении $m$ оценка $\hat{a}=X_1-\frac{5}{6}+\frac{m}{n-1}\sum_{i=2}^n X_i$ параметра $a$ является несмещённой?
 \end{enumerate}
\item Фирма-производитель некоторого лекарственного препарат следит за тем, чтобы концентрация посторонних примесей в препарате в среднем составляла не более 0.03. Для проверки гипотезы о том, что концентрация посторонних примесей равна 0.03, против альтернативной гипотезы о том, что эта концентрация выше 0.03, была взята случайная выборка из 64 образцов препарата. Средняя концентрация примесей в выборке составила 0.0327, а несмещённая оценка дисперсии составила 0.0009. Для проверки гипотезы выбран уровень значимости 10\%.
\begin{enumerate}
\item (3 балла) Рассчитайте статистику, с помощью которой проверяется указанная гипотеза.
\item (2 балла) Рассчитайте критическое значение этой статистики.
\item (2 балла) Выясните, даёт ли выборочное исследование основание считать, что средняя концентрация посторонних примесей превышает допустимый предел в 0.03?
\item (3 балла) Определите, при каких уровнях значимости основная гипотеза будет отвергаться, а при каких --- нет.
\end{enumerate}
\item При 20 наблюдениях с помощью МНК оценивается регрессионное уравнение $y_i=\beta_1+\beta_2 x_i+\beta_3 z_i+\beta_4 t_i+\epsilon_i$ при условиях на ошибки, соответствующих стандартной модели множественной регрессии. Полученные вектор оценок коэффициентов и оценка его матрицы ковариаций равны:\\
$\hat{\beta}=\left[ \begin{array}{c}
9.979620\\
-0.493709\\
0.281451\\
2.955317
\end{array}\right],
\widehat{\Var}(\hat{\beta})=\left[\begin{array}{cccc}
0.264234 & 0.009178 & -0.116760 & -0.264416\\
0.009178 & 0.077753 & 0.014406 & -0.114635\\
-0.116760 & 0.014406 & 0.061492 & 0.098570\\
-0.264416 & -0.116760 & 0.098570 & 0.436001
\end{array}\right]$\\, а оценка дисперсии ошибок регрессии и коэффициент детерминации равны $s^2=0.49367627, R^2=0.832389$.\\ 
Оценивание на тех же данных уравнения $y_i=\gamma_1+\gamma_2 t_i+\epsilon_i$ дало значение коэффициента детерминации $R^2=0.786790$.
\begin{enumerate}
\item (5 баллов) На 5\% уровне значимости тестируйте гипотезу $H_0:\beta_2=0$ против альтернативы $H_a: \beta_2 \ne 0$, а также тестируйте гипотезу $H_0:\beta_3=0$ против альтернативы $H_a: \beta_3 \ne 0$
\item (5 баллов) На 5\%-ном уровне значимости тестируйте гипотезу $H_0:\beta_2=\beta_3=0$ против альтернативы $H_a$: <<не $H_0$>>.
\item (5 баллов) На 5\%-ном уровне значимости тестируйте гипотезу $H_0:\beta_2=\beta_3$ против альтернативы $H_a: \beta_2>\beta_3$.
\end{enumerate}
\end{enumerate}

\subsection{Решение варианта 1, 22.07.2011}
\begin{enumerate}
\item Выписываем функцию Лагранжа: $L=2x-y+3z-\lambda (x^2+y^2+z^2-14)$.\\\\
Условия первого порядка:
$\left\{\begin{aligned}
-2\lambda x+2&=0\\
-2\lambda y-1&=0\\
-2\lambda z+3&=0\\
-x^2-y^2-z^2+14&=0\\
\end{aligned}\right.$\\
Решения системы (критические точки): \\
Точка А: $\left[ x=2,y=-1,z=3,\lambda=1/2\right], f(A)=14$\\
Точка В: $\left[ x=-2,y=1,z=-3,\lambda=-1/2\right], f(B)=-14$\\
Из геометрических соображений очевидно, что одна из точек есть минимум, а другая --- максимум. (График функции есть гиперплоскость, которая ограничивается на сферу).\\
На всякий случай, окаймленная матрица Гессе:
$\left(\begin{array}{cccc}
0 & -2x & -2y & -2z\\
-2x & -2\lambda & 0 & 0\\
-2y & 0 & -2\lambda & 0\\
-2z & 0 & 0 & -2\lambda
\end{array}\right)$\\
На всякий случай, окаймленная матрица Гессе в А: 
$\left(\begin{array}{cccc}
0 & -4 & 2 & -6\\
-4 & -1 & 0 & 0\\
2 & 0 & -1 & 0\\
-6 & 0 & 0 & -1\lambda
\end{array}\right)$.\\
Миноры: $\bigtriangleup_4=-56<0, \bigtriangleup_3=20>0$, максимум.\\
На всякий случай, окаймленная матрица Гессе в В: 
$\left(\begin{array}{cccc}
0 & 4 & -2 & 6\\
4 & 1 & 0 & 0\\
-2 & 0 & 1 & 0\\
6 & 0 & 0 & 1\lambda
\end{array}\right)$.\\
Миноры: $\bigtriangleup_4=-56<0, \bigtriangleup_3=-20<0$, минимум.\\
Баллы:\\
5 баллов за найденные точки\\
5 баллов за обоснование того, что они есть максимум и минимум (с гессианами или без)
\item \begin{enumerate}
\item $I_n=A(3I_n-A)=AB$, т.е. существует обратная матрица $B$, т.е. матрица $A$ невырожденная.
\item Для $n=1$ следует, т.к. из $a^2=0$ следует $a=0$. Для $n>1$ это не верно, например: $A=\left[\begin{array}{cc}
0 & 1\\
0 & 0
\end{array}\right],A^2=
\left[\begin{array}{cc}
0 & 0\\
0 & 0
\end{array}\right]=0.$
\item Рассмотрим базис $\{1,x,x^2,x^3\}$, в этом базисе матрица оператора имеет вид: 
$A=\left[\begin{array}{cccc}
0 & 0 & 0 &0\\
0 & 1 & 0 & 0\\
0 & 0 & 2 & 0\\
0 & 0 & 0 & 3
\end{array}\right]$, т.е. собственные числа 0,1,2,3 и соответствующие собственные векторы $1,x,x^2,x^3$
\end{enumerate}
\item \begin{enumerate}
\item Характеристическое уравнение $(4-\lambda)(4-\lambda)-4=0$, собственные числа: 6,2.\\ Нормированные собственные векторы: $\frac{1}{\sqrt{2}}\left[\begin{array}{c}
1\\
1
\end{array}\right]$ для $\lambda=6$ и $\frac{1}{\sqrt{2}}\left[\begin{array}{c}
1\\
-1
\end{array}\right]$ для $\lambda=2$
\item Собственные числа положительные. Значит, квадратичная форма принимает неотрицательные значения.
\item Матрица А представима в виде $A=CDC^{-1}$, где $C=\frac{1}{\sqrt{2}}\left[\begin{array}{cc}
1 & 1\\
1 & -1
\end{array}\right]$ --- ортогональная матрица, а 
 $D=\left[\begin{array}{cc}
6 & 0\\
0 & 2
\end{array}\right]$ --- диагональная.\\ Значит, $A^n=CD^n C^{-1}$, $D^n=\left[\begin{array}{cc}
6^n & 0\\
0 & 2^n
\end{array}\right]$.
\begin{multline}
{||A^n||}^2=[tr((A^n)^T A^n)]=tr((CD^n C^T)^T CD^n C^T)=tr(CD^n C^T CD^n C^T)=\\
=tr(CD^{2n} C^T)=tr(D^2n C^T C)=tr(D^{2n})=6^{2n}+2^{2n}
\end{multline}

\begin{multline}
\lim_{n\to \infty}\frac{1}{||A^n||}A^n=C\cdot\left(\lim_{n\to \infty}\frac{1}{||A^n||}D^n\right)\cdot C^T=\\
= C\cdot\left(\lim_{n\to \infty}\frac{1}{(6^{2n}+2^{2n})^{1/2}}\right)\left[\begin{array}{cc}
6^n & 0\\
0 & 2^n
\end{array}\right]\cdot C^T=C\cdot\left(\begin{array}{cc}
1 & 0\\
0 & 0
\end{array}\right) \cdot C^T =\\
= \frac{1}{\sqrt{2}}\left[\begin{array}{cc}
1 & 1\\
1 & -1
\end{array}\right]\left(\begin{array}{cc}
1 & 0\\
0 & 0
\end{array}\right)\frac{1}{\sqrt{2}}\left[\begin{array}{cc}
1 & 1\\
1 & -1
\end{array}\right]=\frac{1}{2}\left(\begin{array}{cc}
1 & 1\\
1 & 1
\end{array}\right)
\end{multline}

\end{enumerate}
\item $\lambda^2+4=0 \Rightarrow \lambda_1=2i, \lambda_2=-2i.$ \\
Общее решение дифференциального уравнения имеет вид: $y(x)=C_1\cdot \cos(2x)+C_2\cdot \sin(2x), \forall C_1,C_2 \in \R$.\\
Учитывая $y=0, y'=2$ при $x=0$ получаем $C_1=0, C_2=1$. Функция равна $y(x)=\sin(2x)$, соответственно, $y(2)=\sin(4)\approx 0.757$\\
Баллы:\\
5 баллов за найденное общее решение\\
5 баллов за частное решение и верный ответ.
\item $y(t)=7/9$ является частным решением неоднородного уравнения $y''(t)-8y'(t)-9y(t)+7=0$. Найдем общее решение однородного уравнения $y''(t)-8y'(t)-9y(t)=0$.\\
Характеристическое уравнение $\lambda^2-8\lambda-9=0$ имеет корни $\lambda_1=-1, \lambda_2=9$. Тогда общее решение неоднородного уравнения имеет вид $y(t)=c_1e^{-t}+c_2e^{9t}+\frac{7}{9}$, соответственно,
\begin{multline}
x(t)=\frac{1}{4}(y'(t)-y(t)-1)=\frac{1}{4}\left((c_1e^{-t}+c_2e^{9t}+\frac{7}{9})'-(c_1e^{-t}+c_2e^{9t}+\frac{7}{9})-1\right)= \\
= \frac{1}{4}\left(-c_1e^{-t}+9c_2e^{9t}-c_1e^{-t}-c_2e^{9t}-\frac{7}{9}-1\right)=\frac{1}{4}\left(-2c_1e^{-t}+8c_2e^{9t}-\frac{16}{9}\right)=\\
=\frac{1}{2}\left(-c_1e^{-t}+4c_2e^{9t}-\frac{8}{9}\right)
\end{multline}
Из граничных условий получаем: 
\[\left\{\begin{aligned}
2x(0) &=-c_1+4c_2-\frac{8}{9}&=0\\
2x'(0)&=c_1+36c_2 &=0
\end{aligned}\right.\]
Отсюда получаем 
\begin{equation}
c_1=-\frac{4}{5}, c_2=\frac{1}{45}, y(t)=-\frac{4}{5}e^{-t}+\frac{1}{45}e^{9t}+\frac{7}{9}
\end{equation}
Баллы:\\
5 баллов за найденное общее решение неоднородного уравнения\\
5 баллов за частное решение и верный ответ
\item \begin{enumerate}
\item Обозначим события: $A$ --- <<первый стрелок промахнулся>>, $B$ --- <<второй стрелок промахнулся>>. Тогда событие <<мишень поражена>> можно записать как $\bar{C}=A\cap B$. Искомая вероятность: $\P(C)=1-\P(A\cap B)=1-\P(A)\cdot \P(B)=1-0.3\cdot 0.5=0.85$.
\item Здесь нужно найти условную вероятность события $B$ при условии $C$. По определению условной вероятности, $\P(B\mid C)=\frac{\P(B\cap C)}{\P(C)}$. Совместное наступление событий $B$ и $C$ (второй стрелок промахнулся, но мишень была поражена) эквивалентно тому, что первый стрелок поразил мишень, а второй промахнулся, т.е. $B\cap C=\bar{A}\cap B$. Таким образом, \begin{equation}
P(B\mid C)=\frac{\P(\bar{A}\cap B}{\P(C)}=\frac{\P(\bar{A)\P(B)}}{\P(C)}=\frac{(1-0.3)\cdot 0.5}{0.85}=\frac{0.35}{0.85}\approx 0.4118.
\end{equation}
\end{enumerate}
\item Сначала найдем $c$: $1=F(2)=c\cdot 2^3$, получаем $c=1/8$.
\begin{enumerate}
\item 
\begin{equation}
\P(X<0.3\mid X<0.6)=\frac{\P(X<0.3\cap X<0.6}{\P(C<0.6)}=\frac{\P(X<0.3)}{\P(X<0.6)}=\frac{F(0.3)}{F(0.6)}={\frac{0.3}{0.6}}^3=\frac{1}{8}=0.125
\end{equation}
\item 
\begin{align}
&f(x)=F'(x)=\frac{3}{8}x^2.\\
&EX=\int_0^2 x\frac{3}{8}x^2 dx=\frac{3}{8}\cdot\frac{2^4}{4}=\frac{3}{2}=1.5\\
&\E(\frac{1}{X})=\int_0^2 x^{-1}\frac{3}{8}x^2 dx=\frac{3}{8}\cdot\frac{2^2}{2}=\frac{3}{4}=0.75.\\
&\Cov(X+1,\frac{1}{X})=\Cov(X,\frac{1}{X})=\E(X\cdot \frac{1}{X})-(EX)\E(\frac{1}{X})=1-\frac{3}{2}\cdot \frac{3}{4}=1-\frac{9}{8}=-\frac{1}{8}=-0.125.
\end{align}
\end{enumerate}
\item \begin{enumerate}
\item Допустимое множество значений параметра $a: [0,0.8]$.\\
Найдем математическое ожидание величин $X_i: \E(X_i)=1\cdot a+3\cdot 0.2+5(0.8-a)=4.6-4a$.\\
Приравняем его к выборочному среднему: $4.6-4a=\bar{X}$.\\
Решив полученное уравнение относительно $a$, получаем оценку метода моментов:
\begin{equation}
\hat{a}_{MM}=\frac{4.6-\bar{X}}{4}=1.15-\frac{\bar{X}}{4}
\end{equation}
Эта оценка не может не принадлежать области допустимых значений параметра $a$.
\item Найдём математическое ожидание предложенной оценки: 
\begin{equation}
\E(\hat{a}=m\E(X_1)-1.15+\frac{1}{n-1}\sum_{i=2}^n \E(X_i)=\\=4.6m-4ma-1.15+\frac{1}{n-1}(n-1)(4.6-4a)=(4.6-4a)(m+1)-1.15.
\end{equation}
Оценка $\hat{a}$ будет несмещенной, если $\E(\hat{a})=a$, или $(4.6-4a)(m+1)-1.15=a$. Решив уравнение относительно $m$, получаем \begin{equation}
m=\frac{5a-3.45}{4.6-4a}
\end{equation}
Поскольку $m$, а, следовательно, и $\hat{a}$, зависит от неизвестного параметра, то такого значения $m$ не существует.
\end{enumerate}
\item \begin{enumerate}
\item Тестируем нулевую гипотезу $H_0: \mu_=\mu_0$ против альтернативы $H_A:\mu<\mu_0$ в случае произвольной генеральной совокупности и большого объёма выборки. Для решения этой задачи воспользуемся статистикой $z=\frac{\bar{X}-\mu_0}{s/\sqrt{n}}\sim_{H_0} N(0,1)$. (Можно также предположить, что генеральная совокупность нормальна, и использовать распределение Стьюдента.)\\
\begin{equation}
z=\frac{29.5-32}{\sqrt{36}/\sqrt{49}}=-\frac{5/2}{6/7}=-\frac{35}{12}=-2.917.
\end{equation}
\item Если пользоваться нормальным распределением, то критическое значение $z_{crit}=-z_{0.05}=-1.645$ (Для распределения Стьюдента $t_{crit}=-t_{n-1,\alpha}=-t_{48,0.05}=-1.677.$ Нужного числа степеней свободы в таблице нет, но можно установить, что $t_{crit} \in (-1.684,-1.671).$
\item Так как $z<z_{crit} (z<t_{crit})$, то нулевая гипотеза отвергается, план возмещения убытков стоит пересмотреть.
\item При использовании нормального распределения P-значение=$\P(Z<-2.917)=0.0018$. Таким образом, при уровне значимости выше 0.18\% основная гипотеза будет отвергаться, а при уровне ниже 0.18\% --- не будет. Если пользоваться распределением Стьюдента, то из таблиц можно установить, что P-значение $\in (0.001,0.005)$.
\end{enumerate}
\item \begin{enumerate}

\item 
\begin{equation}
t_{\hat{\beta_2}}=\frac{\hat{\beta}_2}{s_{\hat{\beta}_2}}=\frac{20.8209}{\sqrt{85.97937}}=2.245; t_{\hat{\beta_3}}=\frac{\hat{\beta}_3}{s_{\hat{\beta}_3}}=\frac{1.2309}{\sqrt{19.45426}}=0.279
\end{equation}
поскольку $|t_{\hat{\beta}_2}|>t_{0.25}(16)=2.12$, то гипотеза $H_0: \beta_2=0$ отвергается, соответственно, гипотеза $H_0:\beta_3=0$ не отвергается.
\item 
\begin{equation}
F=\frac{(R_{UR}^2-R_R^2)/q}{(1-R_{UR}^2)/(n-k)}=\frac{(0.243649-0.0035359)/2}{(1-0.243649)/16}=2.54<F_{0.05}(2.16)=3.63
\end{equation}
т.е. гипотеза $\beta_2=\beta_3=0$ не отвергается.
\item Рассмотрим разность $\hat{\beta}_2-\hat{\beta}_3$, оценка её дисперсии равна 
\begin{equation}
\hat{V}(\hat{\beta}_2-\hat{\beta}_3)=\hat{V}(\hat{\beta}_2)+\hat{V}(\hat{\beta}_3)-2\hat{\Cov}(\hat{\beta}_2,\hat{\beta}_3)=85.97937+19.45424-2\cdot8.523841=88.38595
\end{equation}
критическая статистика $t=\frac{\hat{\beta}_2-\hat{\beta}_3}{\sqrt{\hat{V}(\hat{\beta}_2-\hat{\beta}_3)}}$ при нулевой гипотезе имеет распределение $t(16)$. Поскольку $t_{0.05}(16)=1.746$, а $t=2.08>1.746$, то гипотеза $H_0: \beta_2=\beta_3$ отвергается в пользу альтернативы $H_1:\beta_2>\beta_3$
\end{enumerate}
\end{enumerate}

\subsection{Решение варианта 2, 22.07.2011}
\begin{enumerate}
\item Выписываем функцию Лагранжа: $L=x+2y+3z-\lambda (x^2+y^2+z^2-14)$.\\\\
Условия первого порядка:
$\left\{\begin{aligned}
-2\lambda x+1&=0\\
-2\lambda y+2&=0\\
-2\lambda z+3&=0\\
-x^2-y^2-z^2+14&=0\\
\end{aligned}\right.$\\
Решения системы (критические точки): \\
Точка А: $\left[ x=1,y=2,z=3,\lambda=1/2\right], f(A)=14$\\
Точка В: $\left[ x=-1,y=-2,z=-3,\lambda=-1/2\right], f(B)=-14$\\
Из геометрических соображений очевидно, что одна из точек есть минимум, а другая --- максимум. (График функции есть гиперплоскость, которая ограничивается на сферу).\\
На всякий случай, окаймленная матрица Гессе:
$\left(\begin{array}{cccc}
0 & -2x & -2y & -2z\\
-2x & -2\lambda & 0 & 0\\
-2y & 0 & -2\lambda & 0\\
-2z & 0 & 0 & -2\lambda
\end{array}\right)$\\
На всякий случай, окаймленная матрица Гессе в А: 
$\left(\begin{array}{cccc}
0 & -2 & -4 & -6\\
-2 & -1 & 0 & 0\\
-4 & 0 & -1 & 0\\
-6 & 0 & 0 & -1\lambda
\end{array}\right)$.\\
Миноры: $\bigtriangleup_4=-56<0, \bigtriangleup_3=20>0$, максимум.\\
На всякий случай, окаймленная матрица Гессе в В: 
$\left(\begin{array}{cccc}
0 & 2 & 4 & 6\\
2 & 1 & 0 & 0\\
4 & 0 & 1 & 0\\
6 & 0 & 0 & 1\lambda
\end{array}\right)$.\\
Миноры: $\bigtriangleup_4=-56<0, \bigtriangleup_3=-20<0$, минимум.\\
Баллы:\\
5 баллов за найденные точки\\
5 баллов за обоснование того, что они есть максимум и минимум (с гессианами или без)
\item \begin{enumerate}
\item $I_n=A(-2I_n-A)=AB$, т.е. существует обратная матрица $B$, т.е. матрица $A$ невырожденная.
\item Для $n=1$ следует, т.к. из $a^2=a$ следует $a_1=0,a_2=1$. Для $n>1$ это не верно, например: $A=\left[\begin{array}{cc}
0 & 0\\
0 & 1
\end{array}\right],A^2=
\left[\begin{array}{cc}
0 & 0\\
0 & 1
\end{array}\right]$
\item Рассмотрим базис $\{1,x,x^2,x^3\}$, в этом базисе матрица оператора имеет вид: 
$A=\left[\begin{array}{cccc}
0 & 1 & 0 &0\\
0 & 0 & 1 & 0\\
0 & 0 & 0 & 1\\
0 & 0 & 0 & 0
\end{array}\right]$, т.е. собственные числа 0. Cобственные векторы находятся из условия \begin{equation}
Af)(x)=\frac{a_0+a_1 x+a_2 x^2+a_3 x^3-a_0}{x}=a_1+a_2 x+a_3 x^2=0
\end{equation} откуда $a_1=a_2=a_3=0$, т.е. имеется единственный (с точностью до множителя) собственный векторов $f(x)=1$.
\end{enumerate}
\item \begin{enumerate}
\item Характеристическое уравнение $(7-\lambda)(1-\lambda)-16=0$, собственные числа: 9,-1.\\ Cобственные векторы: $\left[\begin{array}{c}
2\\
1
\end{array}\right]$ для $\lambda=9$ и $\left[\begin{array}{c}
1\\
-2
\end{array}\right]$ для $\lambda=-1$
\item Собственные числа разного знака. Значит, квадратичная форма принимает любые значения.
\item Матрица А представима в виде $A=CDC^{-1}$, где $C=\frac{1}{\sqrt{5}}\left[\begin{array}{cc}
2 & 1\\
1 & -2
\end{array}\right]$ --- ортогональная матрица, а 
 $D=\left[\begin{array}{cc}
9 & 0\\
0 & -1
\end{array}\right]$ --- диагональная.\\ Значит, $A^n=CD^n C^{-1}$, $D^n=\left[\begin{array}{cc}
9^n & 0\\
0 & (-1)^n
\end{array}\right]$.\\
\begin{align}
&{||A^n||}^2=[tr((A^n)^T A^n)]=tr((CD^n C^T)^T CD^n C^T)=tr(CD^n C^T CD^n C^T)=tr(CD^{2n} C^T)=tr(D^2n C^T C)=tr(D^{2n})=9^{2n}+1.\\
&\lim_{n\to \infty}\frac{1}{||A^n||}A^n=C\cdot\left(\lim_{n\to \infty}\frac{1}{||A^n||}D^n\right)\cdot C^T=C\cdot\left(lim_{n\to \infty}\frac{1}{(9^{2n}+1)^{1/2}}\right)\left[\begin{array}{cc}
9^n & 0\\
0 & (-1)^n
\end{array}\right]\cdot C^T=C\cdot\left(\begin{array}{cc}
1 & 0\\
0 & 0
\end{array}\right) \cdot C^T = \frac{1}{\sqrt{5}}\left[\begin{array}{cc}
2 & 1\\
1 & -2
\end{array}\right]\left(\begin{array}{cc}
1 & 0\\
0 & 0
\end{array}\right)\frac{1}{\sqrt{5}}\left[\begin{array}{cc}
2 & 1\\
1 & -2
\end{array}\right]=\frac{1}{5}\left(\begin{array}{cc}
4 & 2\\
2 & 1
\end{array}\right)
\end{align}
\end{enumerate}
\item $\lambda^2+3\lambda=0 \Rightarrow \lambda_1=0, \lambda_2=-3.$ \\
Общее решение дифференциального уравнения имеет вид: $y(x)=C_1+C_2\cdot e^{-3x}, \forall C_1,C_2 \in \R$.\\
$0=y(0)=C_1+C_2\cdot e^{-3\cdot 0} \Rightarrow C_1+C_2=0\\
-3=y'(3)=-3C_2\cdot e^{-3\cdot 3}\Rightarrow C_2=e^9$.\\
 Функция равна $y(x)=e^9(-1+e^{-3x})$, соответственно, $y(3)=e^9(-1+e^{3\cdot3})=1-e^9\approx -8102$\\
Баллы:\\
5 баллов за найденное общее решение\\
5 баллов за частное решение и верный ответ.
\item $y(t)=-8/12=-2/3$ является частным решением неоднородного уравнения $y''(t)-8y'(t)+12y(t)+8=0$. Найдем общее решение однородного уравнения $y''(t)-8y'(t)+12y(t)=0$.\\
Характеристическое уравнение $\lambda^2-8\lambda+12=0$ имеет корни $\lambda_1=2, \lambda_2=6$. Тогда общее решение неоднородного уравнения имеет вид $y(t)=c_1e^{2t}+c_2e^{6t}-\frac{2}{3}$, соответственно,
\begin{equation}
x(t)=\frac{1}{2}\left((c_1e^{2t}+c_2e^{6t}-\frac{2}{3})'-4(c_1e^{2t}+c_2e^{6t}-\frac{2}{3})-2\right)=\frac{1}{2}(2c_1e^{2t}+6c_2e^{6t}-4c_1e^{2t}-4c_2e^{6t}+\frac{8}{3}-2)=-c_1e^{2t}+c_2e^{6t}+\frac{1}{3})
\end{equation}
Из граничных условий получаем: $\left\{\begin{aligned}
x(0) &=-c_1+c_2+\frac{1}{3}&=0\\
x'(0)&=-2c_1+6c_2 &=0
\end{aligned}\right.$\\\\
Отсюда получаем 
\begin{equation}
c_1=\frac{1}{2}, c_2=\frac{1}{6}, y(t)=\frac{1}{2}e^{2t}+\frac{1}{6}e^{6t}-\frac{2}{3}
\end{equation}
Баллы:\\
5 баллов за найденное общее решение неоднородного уравнения\\
5 баллов за частное решение и верный ответ
\item \begin{enumerate}
\item Обозначим события: $A$ --- <<первый самолёт поразил цель>>, $B$ --- <<второй самолёт поразил цель>>. Тогда событие <<цель поражена только одним самолётом>> можно записать как $C=(\bar{A}\cap B)\cup (A\cap \bar{B})$. В силу независимости событий $A$ и $B$ искомая вероятность равна:
\begin{equation}
\P(C)=\P(\bar{A})\P(B)+\P(A)\P(\bar{B})=(1-0.6)\cdot 0.4+0.6\cdot (1-0.4)=0.16+0.36=0.52
\end{equation}
\item Здесь нужно найти условную вероятность события $A$ при условии $C$. По определению условной вероятности, $\P(A\mid C)=\frac{\P(A\cap C)}{\P(C)}$. Совместное наступление событий $A$ и $C$ (первый самоёт поразил цель, и цель была поражена только одним самолётом) эквивалентно тому, что первый самолёт поразил цель, а второй --- нет, т.е. $A\cap C=A\cap \bar{B}$. Таким образом,
\begin{equation}
\P(A\mid C)=\frac{\P(A)\cap \bar{B}}{\P(C)}=\frac{\P(A)(1-\P(B))}{\P(C)}=\frac{0.6\cdot 0.6}{0.52}=\frac{0.36}{0.52}\approx 0.6923.
\end{equation}
\end{enumerate}
\item Сначала найдем $c$: $1=F(3)=c\cdot 3^2$, получаем $c=1/9$.
\begin{enumerate}
\item 
\begin{equation}
\P(X>2\mid X>1)=\frac{\P(X>2\cap X>1}{\P(X>1)}=\frac{\P(X>2)}{\P(X>1)}=\frac{1-F(2)}{1-F(1)}=\frac{1-\frac{1}{9}\cdot 4}{1-\frac{1}{9}}=\frac{5}{8}=0.625
\end{equation}
\item 
\begin{align}
& f(x)=F'(x)=\frac{2}{9}x.\\
& EX=\int_0^3 x^2\frac{2}{9}x dx=\frac{2}{9}\cdot\frac{3^4}{4}=\frac{9}{2}=4.5\\
& \E(\frac{1}{X})=\int_0^3 x^{-1}\frac{2}{9}x dx=3\frac{2}{9}=\frac{2}{3}\approx 0.667.\\
& \Cov(X^2+3,\frac{1}{X})=\Cov(X^2,\frac{1}{X})=\E(X^2\cdot \frac{1}{X})-(E(X^2))\E(\frac{1}{X})=2-\frac{9}{2}\cdot \frac{2}{3}=-1.
\end{align}

\end{enumerate}
\item \begin{enumerate}
\item Допустимое множество значений параметра $a: [0,0.7]$.\\
Найдем математическое ожидание величин $X_i: \E(X_i)=1\cdot 0.3+1\cdot a+4(0.7-a)=2.5-3a$.\\
Приравняем его к выборочному среднему: $2.5-3a=\bar{X}$.\\
Решив полученное уравнение относительно $a$, получаем оценку метода моментов:
$\hat{a}_{MM}=\frac{2.5-\bar{X}}{3}$. Эта оценка не может не принадлежать области допустимых значений параметра $a$.
\item Найдём математическое ожидание предложенной оценки:
\begin{equation}
\E(\hat{a}=\E(X_1)-\frac{5}{6}+\frac{m}{n-1}\sum_{i=2}^n \E(X_i)=2.5-3a-\frac{5}{6}+\frac{m}{n-1}(n-1)(2.5-3a)=(2.5-3a)(m+1)-\frac{5}{6}=a
\end{equation}
Оценка $\hat{a}$ будет несмещенной, если $\E(\hat{a})=a$, или $(2.5-3a)(m+1)-\frac{5}{6}=a$. Решив уравнение относительно $m$, получаем 
\begin{equation}
m=\frac{5a-3.45}{4.6-4a}
\end{equation}
Поскольку $m$, а, следовательно, и $\hat{a}$, зависит от неизвестного параметра, то такого значения $m$ не существует.
\end{enumerate}
\item \begin{enumerate}
\item Тестируем нулевую гипотезу $H_0: \mu_=\mu_0$ против альтернативы $H_A:\mu<\mu_0$ в случае произвольной генеральной совокупности и большого объёма выборки. Для решения этой задачи воспользуемся статистикой $z=\frac{\bar{X}-\mu_0}{s/\sqrt{n}}\sim_{H_0}N(0,1)$. (Можно также предположить, что генеральная совокупность нормальна, и использовать распределение Стьюдента.)\begin{equation}
z=\frac{0.0327-0.03}{\sqrt{0.0009}/\sqrt{64}}=-\frac{0.0027}{0.03/8}=0.72.
\end{equation}
\item Если пользоваться нормальным распределением, то критическое значение $z_{crit}=-z_{0.1}=1.28$ (Для распределения Стьюдента $t_{crit}=-t_{n-1,\alpha}=-t_{63,0.1}=1.295.$ Нужного числа степеней свободы в таблице нет, но можно установить, что $t_{crit} \in (1.289,1.296).$
\item Так как $z<z_{crit} (z<t_{crit})$, то нет оснований отвергать основную гипотезу и считать, что допустимый предел концентрации превышен.
\item При использовании нормального распределения P-значение=$\P(Z>0.72)=0.2358$. Таким образом, при уровне значимости выше 23.58\% основная гипотеза будет отвергаться, а при уровне ниже 23.58\% --- не будет. Если пользоваться распределением Стьюдента, то из таблиц можно установить, что P-значение $\in (0.2,0.25)$.
\end{enumerate}
\item \begin{enumerate}

\item \begin{equation}
t_{\hat{\beta_2}}=\frac{\hat{\beta}_2}{s_{\hat{\beta}_2}}=\frac{-0.49371}{\sqrt{0.077753}}=-1.77; t_{\hat{\beta_3}}=\frac{\hat{\beta}_3}{s_{\hat{\beta}_3}}=\frac{0.281451}{\sqrt{0.061492}}=1.13
\end{equation}
поскольку $|t_{\hat{\beta}_2}|<t_{0.25}(16)=2.12$, то гипотеза $H_0: \beta_2=0$ не отвергается, соответственно, гипотеза $H_0:\beta_3=0$ не отвергается.
\item \begin{equation}
F=\frac{(R_{UR}^2-R_R^2)/q}{(1-R_{UR}^2)/(n-k)}=\frac{(0.832389-0.786790)/2}{(1-0.832389)/16}=2.18<F_{0.05}(2.16)=3.63
\end{equation} т.е. гипотеза $\beta_2=\beta_3=0$ не отвергается.
\item Рассмотрим разность $\hat{\beta}_2-\hat{\beta}_3$, оценка её дисперсии равна \begin{equation}
{V}(\hat{\beta}_2-\hat{\beta}_3)=\hat{V}(\hat{\beta}_2)+\hat{V}(\hat{\beta}_3)-2\hat{\Cov}(\hat{\beta}_2,\hat{\beta}_3)=0.077753+0.061492-2\cdot 0.014406=0.110433
\end{equation} критическая статистика $t=\frac{\hat{\beta}_2-\hat{\beta}_3}{\sqrt{\hat{V}(\hat{\beta}_2-\hat{\beta}_3)}}$ при нулевой гипотезе имеет распределение $t(16)$. Поскольку $t_{0.05}(16)=1.746$, а $t=-2.33<-1.746$, то гипотеза $H_0: \beta_2=\beta_3$ отвергается в пользу альтернативы $H_1:\beta_2<\beta_3$
\end{enumerate}
\end{enumerate}


\section{2012}
\subsection{Вступительная олимпиада в магистратуру}
Во всех задачах: штраф за арифметическую ошибку -1 балл. Вес каждой задачи --- 10 баллов.
\begin{enumerate}
\item Вычислить предел выражения $\frac{n+x}{n-1}^n$ при $n\to \infty$.\\
\item Рассмотрите следующую систему линейных алгебраических уравнений.\\
$\left\{
\begin{aligned}
3x_1+5x_2-4x_3+2x_4 &=0\\
2x_1+4x_2-6x_3+3x_2 &=0\\
11x_1+17x_2-8x_3+4x_4&=0.
\end{aligned}\right.$\\
\begin{enumerate}
\item Найдите размерность пространства решённой системы.
\item Найдите фундаментальную систему решений системы.
\item Опишите общее решение системы через фундаментальную систему решений.
\end{enumerate}
\item В пространстве $\R^4$ со стандартным скалярным произведением найти ортогональную проекцию $g$ и перпендикуляр $h$, опущенный из вектора $f=(7,-4,-1,2)$ на подпространство $L$, заданное однородной системой уравнений 
$\left\{\begin{aligned}
2x_1+x_2+x_3+3x_4 &=0\\
3x_1+2x_2+2x_3+x_4 &=0\\
x_1+2x_2+2x_3-9x_4 &=0.
\end{aligned}\right.$\\
\item Найдите решение дифференциального уравнения $y^{(4)}+y^{(2)}=2\cos x$, удовлетворяюшее начальным условиям $y(0)=-2, y'(0)=1, y''(0)=y'''(0)=0$.\\
\item Найдите критические точки для функции
\begin{equation}
f(x,y)=x+xy+y^2-4\ln x-10\ln y
\end{equation}
и укажите их тип.\\
\item В задаче поиска экстремумов функции 
\begin{equation}
f(x,y,z)=x^2+y^2+z^2+xyz
\end{equation}
при ограничении $x+y+z=4$
\begin{enumerate}
\item Найдите все критические точки, удовлетворяющие необходимым условиям первого порядка
\item для всех критических точках, найденных в пункте (А),выпишите какие-либо достаточные условия второго порядка и проведите классификацию критических точек.
\end{enumerate}
\item Контрольная работа состоит из тестовых вопросов. В каждом вопросе 5 вариантов ответа из которых один – верный. Студент не очень готовился и поэтому знает 40\% вопросов. Неправильный ответ на тесте не штрафуется, поэтому если студент не знает ответа, то он отвечает наугад равновероятно.
\begin{enumerate}
\item Какова вероятность того, что студент отметит верный ответ на первый вопрос теста?
\item Какова вероятность того, что студент знает ответ на первый вопрос теста, если он отметил верный ответ?
\item Какова вероятность того, что студент знает ответ хотя бы на один вопрос из первых двух, если он отметил верные ответы на оба из них?
\end{enumerate}
\item В ходе анкетирования 225 человек ответили на вопрос о том, сколько времени они проводят на работе ежедневно. Среднее выборочное оказалось равно 9.5 часов при выборочном стандартном отклонении 0.6 часа.
\begin{enumerate}
\item Постройте 90\% доверительный интервал для математического ожидания времени, проводимого на работе
\item Проверьте гипотезу о том, что в среднем люди проводят на работе 9 часов, против альтернативной гипотезе о том, что в среднем люди проводят на работе больше 9 часов, укажите точное Р-значение.
\item Сформулируйте предпосылки, которые были использованы для проведения теста.
\end{enumerate}
\item Наблюдения $X_1,X_2,...,X_n$ независимы и одинаково распределены с функцией плотности $f(x)=\frac{a(\ln(x))^{a-1}}{x}$ при $x\in [1;\e]$. По 100 наблюдениям известно, что $\sum_{i=1}^100 \ln(\ln(X_i))=-20$
\begin{enumerate}
\item Постройте оценку $\hat{a}$ для неизвестного параметра $a$ методом максимального правдоподобия.
\item Найдите наблюдаемую информацию Фишера.
\item Постройте 90\% доверительный интервал для $a$
\end{enumerate}
\item Исследователь располагает следующими данными по 935 респондентам (U.S. NLS80 Database, J. Wooldridge):\\
\begin{itemize}
\item ежемесячный доход, EARNINGS, долл.;
\item возраст, AGE, число лет;
\item число лет обучения, EDUC;
\item расовая принадлежность, BLACK = 1, если респондент – афроамериканец, = 0 в противном случае;
\item URBAN = 1, если респондент проживает в крупном городе.
\end{itemize}
С помощью метода наименьших квадратов он оценивает следующие регрессии, в каждой из которых LWAGE = LN(WAGE) является зависимой переменной, а объясняющие переменные:
\begin{enumerate}
\item[(1)] AGE, EDUC, URBAN, по всей выборке
\item[(2)] AGE, EDUC, URBAN, BLACK, по всей выборке
\item[(3)] AGE, EDUC, URBAN, только для ‘whites’
\item[(4)] AGE, EDUC, URBAN, только для blacks
\item[(5)] AGE, EDUC, TENURE, BLACK, AGE*BLACK, EDUC*BLACK, URBAN*BLACK, по всей выборке.
\end{enumerate}

Результаты оценивания представлены в таблице 1. RSS = сумма квадратов остатков. В скобках приведены стандартные ошибки. К сожалению, некоторые результаты оценивания для Модели 4 в таблице были по ошибке не внесены исследователем в таблицу.
\begin{enumerate}
\item Вычислите пропущенные в Модели 4 оценки коэффициентов и RSS. Поясните Ваши вычисления.
\item Дайте содержательную интерпретацию коэффициентам при BLACK, AGE*BLACK и URBAN*BLACK в Модели 2.
\item Проверьте совместную значимость коэффициентов при переменных BLACK, AGE*BLACK, EDUC*BLACK, TENURE*BLACK, URBAN*BLACK в Модели 5.
\item Объясните, как связан тест из пункта (c) с тестом Чоу для Моделей 1, 3 и 4.
\item Объясните, достаточно ли проверить на значимость коэффициент при BLACK в Модели 2, чтобы показать наличие дискриминации по расовой принадлежности.
\end{enumerate}
Таблица 1.\\
$\begin{tiny}
\begin{tabular}{|c|ccccc|}
 \hline
 & 1 & 2 & 3 & 4 & 5\\
 COEFFICIENT & LWAGE & LWAGE & LWAGE & LWAGE & LWAGE\\
 AGE & 0.0163(0.00417) & 0.0159(0.00409) & 0.0185(0.00440) & ? & 0.0185(0.00438)\\
 EDUC & 0.0587(0.00569) & 0.0522(0.00569) & 0.0541(0.00593) & ? & 0.0541(0.00592)\\
 URBAN & 0.177(0.0278) & 0.191(0.0273) & 0.196(0.0288) & ? & 0.196(0.0287)\\
 BLACK & & -0.228(0.0374) & & & 0.671(0.504)\\
 AGE*BLACK & & & & & -0.0194(0.0121)  \\
 EDUC*BLACK & & & & & -0.0296(0.0209)\\
 URBAN*BLACK & & & & & -0.0381(0.0903)\\
 CONSTANT & 5.219(0.156) & 5.348(0.154) & 5.251(0.163) & ? & 5.251(0.163)\\
 \hline
 Observations & 935 & 935 & 815 & 815 & 935\\
 R-squared & 0.185 & 0.217 & 0.183 & 0.183 & 0.226\\
 RSS & 134.930 & 129.752 & 112.948& ? & 128.185\\
 \hline
\end{tabular}
\end{tiny}$
\end{enumerate}

\subsection{Решение вступительной олимпиады}
\begin{enumerate}
\item Находим решение с помощью второго замечательного предела. При $n \to \infty$
\begin{equation}
\lim(\frac{n+x}{n-1}^n=\lim (1+\frac{x+1}{n-1})^n=\begin{cases}
\lim (1+\frac{x+1}{n-1})\cdot \left(\lim(1+\frac{x+1}{n-1})^{\frac{n-1}{x+1}}\right)^{x+1},   \text{при} x\ne -1  \\
1, \text{при} x=-1
\end{cases}
\end{equation}
Ответ: $e^{x+1}$.\\\\
\item 
$\left[ \begin{array}{cccc|c}
3 & 5 & -4 & 2 & 0\\
2 & 4 & -6 & 3 & 0\\
11 & 17 & -8 & 4 & 0
\end{array}\right] \to
\left[ \begin{array}{cccc|c}
3 & 5 & -4 & 2 & 0\\
2 & 4 & -6 & 3 & 0\\
0 & -2 & 10 & -5 & 0
\end{array}\right] \to
\left[ \begin{array}{cccc|c}
1 & 1 & 2 & -1 & 0\\
2 & 4 & -6 & 3 & 0\\
0 & -2 & 10 & -5 & 0
\end{array}\right] \to\\\\
\left[ \begin{array}{cccc|c}
1 & 1 & 2 & -1 & 0\\
0 & 2 & -10 & 5 & 0\\
0 & -2 & 10 & -5 & 0
\end{array}\right] \to
\left[ \begin{array}{cccc|c}
1 & 1 & 2 & -1 & 0\\
0 & 2 & -10 & 5 & 0
\end{array}\right]$.\\\\
Следовательно, $rg(A)=2$, где $A$ --- основная матрица системы.\\
 Известно, что $\dim L=n-rg(A)=4-2=2$, где $L$ --- множество решений линейной однородной системы, $n$ --- число переменных системы.\\
 Фундаментальная система решений: $e_1=\left(\begin{array}{cccc}
 7 & 5 & 1 & 0
 \end{array}\right)^T$ и $e_2=\left(\begin{array}{cccc}
 7 & -5 & 0 & 2
 \end{array}\right)^T$.\\
 Общее решение системы: $L={x=\alpha_1 \cdot \e_1+\alpha_2\cdot \e_2:\forall \alpha_1,\alpha_2 \in \R}$.\\\\
 \textbf{Критерии:}\\
 \begin{enumerate}
 \item Найдите размерность пространства решений системы. \textbf{2 балла}
 \item Найдите фундаментальную систему решений системы. \textbf{3 балла}
 \item Опишите общее решение системы через фундаментальную систему решений. \textbf{5 баллов}
 \end{enumerate}
 Если решение системы найдено без фундаментальной системы - \textbf{7 баллов}\\\\
 \item $\left[ \begin{array}{cccc|c}
2 & 1 & 1 & 3 & 0\\
3 & 2 & 2 & 1 & 0\\
1 & 2 & 2 & -9 & 0
\end{array}\right] \to
\left[ \begin{array}{cccc|c}
2 & 1 & 1 & 3 & 0\\
0 & -1 & -1 & 7 & 0\\
1 & 2 & 2 & -9 & 0
\end{array}\right] \to
\left[ \begin{array}{cccc|c}
0 & -3 & -3 & 21 & 0\\
0 & -1 & -1 & 7 & 0\\
1 & 2 & 2 & -9 & 0
\end{array}\right] \to\\\\
\left[ \begin{array}{cccc|c}
1 & 2 & 2 & -9 & 0\\
0 & 1 & 1 & -7 & 0
\end{array}\right] \to
\left[ \begin{array}{cccc|c}
1 & 0 & 0 & 5 & 0\\
0 & 1 & 1 & -7 & 0
\end{array}\right]$.\\\\
Фундаментальная система решений: $e_1=\left(\begin{array}{cccc}
0 & -1 & 1 & 0
\end{array}\right)^T$ и $e_1=\left(\begin{array}{cccc}
-5 & 7 & 0 & 1
\end{array}\right)^T$.\\
Следовательно, $L=L(e_1,e_2).\\
f=g+h\\
f=\alpha_1 \cdot \e_1+\alpha_2 \cdot \e_2 +h;
\left\{ \begin{aligned}
(e_1,f)&=\alpha_1 \cdot (e_1,e_1)+\alpha_2\cdot (e_1,e_2)+(e_1,h),\\
(e_2,f)&=\alpha_1 \cdot (e_2,e_1)+\alpha_2\cdot (e_2,e_2)+(e_2,h)
\end{aligned}\right.\\
\left\{ \begin{aligned}
(e_1,f)&=\alpha_1 \cdot (e_1,e_1)+\alpha_2\cdot (e_1,e_2),\\
(e_2,f)&=\alpha_1 \cdot (e_2,e_1)+\alpha_2\cdot (e_2,e_2)
\end{aligned}\right.\\
\left\{ \begin{aligned}
3&=\alpha_1 \cdot 2+\alpha_2\cdot (-7)\\
-61&=\alpha_1 \cdot (-7)+\alpha_2\cdot 75
\end{aligned}\right.\\
\alpha_1=-2, \alpha_2=-1.\\
g=\alpha_1 \cdot \e_1+\alpha_2 \cdot \e_2=-2 \left(\begin{array}{cccc}
0 & -1 & 1 & 0
\end{array}\right)^T-1\left(\begin{array}{cccc}
-5 & 7 & 0 & 1
\end{array}\right)^T=\left(\begin{array}{cccc}
5 & -5 & -2 & -1
\end{array}\right)^T;\\
h=f-g=\left(\begin{array}{cccc}
-7 & -4 & -1 & 2
\end{array}\right)^T-\left(\begin{array}{cccc}
5 & -5 & -2 & -1
\end{array}\right)^T=\left(\begin{array}{cccc}
2 & 1 & 1 & 3
\end{array}\right)^T.$\\\\
\textbf{Критерии}:
\begin{itemize}
\item Найдена только фундаментальная система решений --- \textbf{3 балла}
\item Записана, но не решена система уравнений для определения коэффициентов разложения проекции по базисным векторам --- \textbf{5 баллов}
\item Найдена проекция, не найден перпендикуляр --- \textbf{7 баллов}
\item Найдены проекция и перпендикуляр --- \textbf{10 баллов}
\end{itemize}
\smallskip
\item Составим характеристическое уравнение, соответствующее однородному дифференциальному уравнению: $\lambda^4+\lambda^2=0$.
\smallskip
$\lambda_1=\lambda_2=0, \lambda_3=i,\lambda_4=-i$ --- решения характеристического уравнения.
Согласно общей теории линейных уравнений с постоянными коэффициентами имеем 
\begin{equation}
y_{\text{общ.одн.}}(x)=C_1+C_2\cdot x+C_3\cdot \cos x+C_4\cdot \sin x
\end{equation} где $C_1,C_2,C_3,C_4$ --- произвольные вещественные числа.\\
Частное решение неоднородного дифференциального уравнения будем искать в виде:
\begin{equation}
y_{\text{част.неодн.}}(x)=x(A \cos x+B\cdot \sin x
\end{equation}
Подставляя частное решение с неопределёнными коэффициентами в неоднородное дифференциальное уравнение, получаем, что $A=0, B=-1$. \\
Следовательно, 
\begin{equation}
y_{\text{част.неодн.}}(x)=-x\sin x.
\end{equation}
Согласно теории решения неоднородных дифференциальных уравнений с постоянными коэффициентами имеем:
\begin{equation}
y_{\text{общ.неодн.}}(x)=y_{\text{общ.одн.}}(x)+y_{\text{част.неодн.}}(x)
\end{equation}
Значит,
\begin{equation}
y_{\text{общ.неодн.}}(x)=C_1+C_2\cdot x+C_3\cdot \cos x+C_4\cdot \sin x-x\sin x
\end{equation} где $C_1,C_2,C_3,C_4$ --- произвольные вещественные числа.\\
Теперь общее решение для неоднородного дифференциального уравнения известно. Находим частное решение, которое удовлетворяет начальным условиям. В итоге получаем, что $C_1=0, C_2=1, C_3=-2, C_4=0$. Значит,
\begin{equation}
y(x)=x-2\cos x-x\sin x
\end{equation}
\textbf{Критерии}
\begin{enumerate}
\item Выписано для однородного уравнения общее решение с константами --- \textbf{5 баллов}
\item Выписан общий вид частного решения для неоднородного уравнения --- \textbf{7 баллов}
\item Найдены константы --- \textbf{10 баллов}
\end{enumerate}
\medskip
\item Находим условия первого порядка:
$\left\{\begin{aligned}
y-\frac{4}{x}+1 & =0\\
x+2y-\frac{10}{y} & =0
\end{aligned}\right.$\\
Решение системы сводится к решению кубического уравнения:
\begin{equation}
x^2(x-4)-2(x-4)^2+10x^2=0
\end{equation}
Находим корень уравнения $x\approx 1.4$\\
Получаем единственную критическую точку $x\approx 1.4, y\approx 1.9$\\
Находим матрицу Гессе и знак угловых миноров
$\left(\begin{array}{cc}
4/x^2 & 1\\
1 & 2+10/y^2
\end{array}\right)$\\
Знаки: $\bigtriangleup_1>0, \bigtriangleup_2>0$ $\Rightarrow$ найденная точка --- локальный минимум.\\
\item Запишем функцию Лагранжа:
\begin{equation}
L(x,y,z,\lambda)=x^2+y^2+z^2+xyz-\lambda(x+y+z-4)
\end{equation}
\begin{enumerate}
\item Критически точки являются решениями системы уравнений\\
$\left\{ \begin{aligned}
\frac{\partial L}{\partial x}& =2x-\lambda yz & =0\\
\frac{\partial L}{\partial y}& =2y-\lambda xz & =0\\
\frac{\partial L}{\partial z}& =2z-\lambda xy & =0\\
\frac{\partial L}{\partial \lambda}& =4-x-y-z & =0
\end{aligned}\right.$\\
Находим четыре точки: $M_1(2,2,0), M_2(2,0,2), M_3(0,2,2),M_4(\frac{4}{3},\frac{4}{3},\frac{4}{3})$\\
\item Достаточные условия второго порядка --- знакоопределенность второго дифференциала функции $f(x,y,z)$ при ограничении на дифференциалы $dx+dy+dz=0$:
\begin{eqnarray}
&d^2 f=2dx^2+2d y^2+2d z^2+2zdxdy+2ydxdz+2xdydz=[\textit{при выполнении }  dx+dy+dz=0]=\\
&=(4-2y)dx^2-(4-2x)dy^2+2(z-x-y+2)dxdy=A(x,y)
\end{eqnarray}
Для точек $M_1(2,2,0), M_2(2,0,2), M_3(0,2,2)$ квадратичная форма $A(x,y)$ не является знакоопределенной, так как
\begin{equation}
\det \left(\begin{array}{cc}
4-2y & 2+z-x-y\\
2+z-x-y & 4-2x
\end{array}\right) <0
\end{equation}
В точке $M_4(\frac{4}{3},\frac{4}{3},\frac{4}{3})$ квадратичная форма $A(x,y)$ по критерию Сильвестра отрицательно определена, так как первый и второй миноры её матрицы положительны:
\begin{equation}
\det \left(\begin{array}{cc}
\frac{4}{3} & \frac{2}{3}\\\\
\frac{2}{3} & \frac{4}{3}
\end{array}\right)>0
\end{equation}
\end{enumerate}
Следовательно, точка $M_4(\frac{4}{3},\frac{4}{3},\frac{4}{3})$ является точкой минимума.\\
\textbf{Критерии:}\\
Вес каждого пункта задания --- 5 баллов.
\item Обозначим события:\\
$A_1$ --- студент знает ответ на 1-ый вопрос\\
$B_1$ --- студент выбрал верный ответ на 1-ый вопрос\\
$A_2, B_2$ --- аналогично для 2-го вопроса.\\
\begin{enumerate}
\item $\P(B_1)=\P(A_1)+0.2(1-\P(A_1))=0.4+0.2\cdot 0.6=0.52$
\item $\P(A_1\mid B_1)=\P(A_1\cap B_1)/\P(B_1)=0.4/0.52=10/ 0.77$
\item $\P(A_1\cup A_2)\mid B_1\cap B_2)=1-\P(\bar{A_1}\cap \bar{A_2}\mid B_1\cap B_2)=1-\P(\bar{A_1}\mid B_1)\P(\bar{A_2}\mid B_2)=1-(3/13)^2=160/169\approx 0.95$
\end{enumerate}
Некоторые студенты трактовали условие по-другому: Студент знает q вопросов из общего количества в N вопросов, причем  q/N=0.4. Такая трактовка тоже засчитывалась как верная. При этом вероятности в первых двух пунктах не меняются, а вероятность для третьего будет зависеть от N.\\
\textbf{Критерии}
\begin{enumerate}
\item 3 балла
\item 3 балла
\item 4 балла
\end{enumerate}
Всего 10 баллов.
\item 
\begin{enumerate}
\item Общий вид доверительного интервала: 
\begin{equation}
\left[ \bar{X}-z\frac{s}{\sqrt{n}};\bar{X}+z\frac{s}{\sqrt{n}}\right]
\end{equation}
По условию, $\bar{X}=9.5, s=0.6, n=225$.\\
Квантиль нормального распределения для 90\% доверительного интервала $z=1.65$
Тогда численное значение доверительного интервала: $[9.434;9.566]$
\item Тестирование гипотезы\\
\begin{equation}
Z=\frac{\bar{X}-\mu}{s/\sqrt{n}}=\frac{(9.5-9)\cdot 15}{0.6}=12.5
\end{equation}
Значение 12.5 фантастически велико для нормального распределеия, поэтому P-значение равно нулю. При любом разумном уровне значимости, будь то $\alpha=0.01, \alpha=0.05$ или $\alpha=0.1$ гипотеза  $H_0$ отвергается.
\item Предпосылки:
 все $X_i$ независимы, одинаково распределены и имеют конечное матиматическое ожидани и дисперсию.
 \end{enumerate}
 \textbf{Критерии}
 \begin{enumerate}
 \item 4 балла
 \item 3 балла
 \item 3 балла
 \end{enumerate}
 В сумме 10 баллов.
 \item \begin{enumerate}
 \item Функция правдоподобия: 
\begin{equation}
 L(a)=\prod_i \frac{a \ln^{a-1}x_i}{x_i}
\end{equation} 
 Лог-функция правдоподобия: 
\begin{equation}
 l(a)=\sum_i \ln a+(a-1)\ln\ln x_i-\ln x_i
\end{equation} 
 Производная: 
\begin{equation}
l'(a)=\sum \frac{1}{a}+\ln\ln x_i=\frac{n}{a}-\ln\ln x_i
\end{equation} 
 Условие первого порядка: 
\begin{equation}
\hat{a}=-\frac{n}{\sum \ln\ln x_i}=\frac{100}{20}=5
\end{equation} 
 Проверяем, что нашли максимум: $l''(a)=-\frac{n}{a^2}<0$ $\Rightarrow$ действительно нашли максимум.
 \item Наблюдаемая информация Фишера: $\hat{I}=-l''(\hat{a})=\frac{n}{\hat{a}^2}=4$
 \item Оценка дисперсии: $\hat{\Var}(\hat{a})=\hat{I}^2=0.25$\\
 Поскольку оценки максимального правдоподобия асимптотические нормальны и несмещены, доверительный интервал в общем виде:
 \begin{equation}
 \left[\hat{a}-z\sqrt{\Var(\hat{a}};\hat{a}-z\sqrt{\Var(\hat{a}}\right]
 \end{equation}
 Численно: $[4.18;5.82]$
 \end{enumerate}
 \textbf{Критерии}\\
 \begin{enumerate}
 \item 4 балла
 \item 3 балла
 \item 3 балла
 \end{enumerate}
 Всего 10 баллов.
 \item При проверке данной задачи комиссией было принято решение проверять пункты (a), (b) и (d), на ход решения которых не влияли опечатки.
 \begin{itemize}
 \item[(a)] Коэффициенты можно найти, используя результаты оценивания модели 5:\\
 $
\begin{tabular}{|c|cc|}
 \hline
 & 4 & 5\\
 COEFFICIENT & LWAGE & LWAGE \\
 AGE & 0.0185-0.0194=0.0009 & 0.0185\\
 EDUC & 0.0541-0.0296=0.0245 & 0.0541\\
 URBAN & 0.196-0.0381=0.1579 & 0.196\\
 BLACK & & 0.671\\
 AGE*BLACK & & -0.0194 \\
 EDUC*BLACK & & -0.0296\\
 URBAN*BLACK & & -0.0381\\
 CONSTANT & 5.251+0.671=5.922 & 5.251\\
 RSS & 128.185-112.948=112.948 & 128.185\\
 \hline
\end{tabular}
$\\
RSS можно найти как разницу между RSS в модели 5(с учетом дамми-переменной black, характеризующей различия в коэффициентах между white и black) и RSS в модели 3 (только для white): $RSS_4=RSS_5-RSS_3$.\\
Аналогично можно было вывести через TSS по всей выборке (из моделей 1, 2 или 5) и TSS для модели 3(по white):
\begin{equation}
RSS_4=(1-R_4^2(TSS_5-TSS_3)
\end{equation}
где $TSS=\frac{RSS_5}{(1-R_5^2)}, TSS_3=\frac{RSS_3}{(1-R_3^2)}$.
\item[(b)] Коэффициент значим (при любом разумном уровне значимости): $t=\frac{-0.228}{0.0374}=-6.1$.\\
У афроамериканцев заработная плата ниже (при прочих равных) на 20.4: \% ($\exp(-0.228)-1)\cdot 100\%=-20.4\% $(для полулогарифмической модели).\\
Приближенная интерпретация: на 22.8\%, т.е. $ -0.228 \cdot 100 \%=-22.8\% $
\item[(d)] Недостаточно: если коэффициент незначим, это еще не означает отсутствия проблемы дискриминации, если значим --- не факт, что спецификация модели верна (могут быть пропущены существенные переменные; дискриминация проявляется не только <<в среднем>>, но и в отдаче на другие регрессоры модели).
 \end{itemize}
 \textbf{Критерии:}
 \begin{itemize}
 \item[A] 0.5 баллов за каждый коэффициент(итого 2)+3 балла за RSS=5 баллов
 \item[B] 2 балла за интерпретацию коэффициента при переменной BLACK в модели 2.
 \item[D] 3 балла
 \end{itemize}
\end{enumerate}

\section{2013}

\subsection{Демо-версия олимпиады 2013}

\begin{enumerate}

\item Вычислить предел 
\[
\lim_{n\to\infty} \frac{\sqrt{4n^2+5n+1}-2n}{e^{\cos(\arctg n)}}
\]
\item Найти собственные числа и собственные векторы матрицы
\[
\left(
\begin{array}{ccc}
4 & 5 & 2 \\
5 & 7 & 3 \\
6 & 9 & 4
\end{array}
\right)
\]
\item Исследуйте на экстремум функцию $F(x,y)=4x^3+10x^2+2y^2+2xy^2+9$

\item Пусть $F(x,y)=9-x^2-y^2$ при ограничении $a+bx+cy=0$, $a\neq 0$, $b\neq 0$, $c\neq 0$. При каких значениях параметров
ограничения множество условных локальных экстремумов функции $F(x,y)$ будет не
пусто? Каков характер этих экстремумов?

\item  Найдите частное решение дифференциального уравнения 
\[
y'=\frac{y^5+3x^2\cos(y)}{x^3\sin(y)-3y^2-5y^4x}
\]
удовлетворяющее начальному условию $y(1)=0$.

\item Погода завтра может быть ясной с вероятностью 0.3 и пасмурной с вероятностью 0.7.
Вне зависимости от того, какая будет погода, Маша даёт верный прогноз с вероятностью
0.8. Вовочка, не разбираясь в погоде, делает свой прогноз по принципу: с вероятностью
0.9 копирует Машин прогноз, и с вероятностью 0.1 меняет его на противоположный.
\begin{enumerate}
\item Какова вероятность того, что Машин и Вовочкин прогнозы совпадут?
\item Какова вероятность того, что Маша спрогнозирует ясный день?
\item Какова вероятность того, что день будет ясный, если Маша спрогнозировала
ясный?
\item Какова вероятность того, что день будет ясный, если Вовочка спрогнозировал
ясный?
\end{enumerate}


\item Для того чтобы поступить в университет, абитуриенту Васе Смирнову необходимо
сдать два экзамена: по математике и по английскому языку. Экзамен по математике
оценивается по десятибалльной шкале, а экзамен по английскому языку по пятибалльной.
Предполагается, что шкалы оценок непрерывные, например, на экзамене по математике
абитуриент может получить $4.734(34)\ldots$ балла. Известно, что для поступления на бюджет
необходимо набрать 11 из 15 баллов. 


Кроме того, необходимо получить по математике не
ниже 4 баллов, а по английскому языку не ниже 3 баллов для того, чтобы участвовать в
конкурсе на бюджетные места. Функция совместной плотности распределения
вероятности получения определенной оценки по математике, $X$, и по английскому языку, $Y$, для Васи имеет следующий вид:
\[
f(x,y)=
\begin{cases}
axy, \, 0\leq x\leq 100,\, 0\leq y\leq 5 \\
0,\, \text{иначе}
\end{cases}
\]
\begin{enumerate}
\item Определите значение параметра $a$, при котором указанная
функция может являться функцией плотности. 
\item Найдите вероятность того, что Вася
поступит на бюджет. 
\item Как изменится вероятность поступления на бюджет, если известно,
что за экзамен по английскому Вася получил 4 балла?
\end{enumerate}


\item Известно, что случайная величина $X$ распределена равномерно на отрезке $[0; a]$.
Исследователь проверяет гипотезу $H_0$: $a=10$ против $H_a$: $a>10$ c помощью
следующего критерия: отвергнуть $H_0$ в пользу $H_a$, если $X<c$. Каким должно быть
число $c$, если исследователь хочет осуществить проверку на уровне значимости 10\%? При
$c=8$ выразите мощность критерия как функцию от $a$.

\item Статистик Тимофей оценивает доверительный интервал для математического ожидания по большой выборке по формуле
\[
\bar{X}-z_{\alpha/2}\frac{\hat{\sigma}}{\sqrt{n}} < \mu < \bar{X}+z_{\alpha/2}\frac{\hat{\sigma}}{\sqrt{n}}
\]
Тимофей забыл таблицы нормального распределения и не может точно вспомнить значение $z_{\alpha/2}$ для уровня
доверия (доверительной вероятности) 95\%. Определите, каков будет уровень доверия, если
\begin{enumerate}
\item Тимофей подставит значение $z_{\alpha/2}=2$
\item Тимофей воспользуется следующим выражением для доверительного интервала:
\[
\bar{X}-1.5\frac{\hat{\sigma}}{\sqrt{n}} < \mu < \bar{X}+2.5\frac{\hat{\sigma}}{\sqrt{n}}
\]
\end{enumerate}


\item Посредник, торгующий подержанными автомобилями, для получения данных о
предложениях продажи пользуется журналом, где публикуются цены предложения (Price),
возраст автомобиля (Age), его пробег (Run), наличие сигнализации (Signal) и музыкальной
системы (Music). Посреднику необходимо решить две проблемы.

\begin{enumerate}
\item[Проблема 1.] У посредника сложилось впечатление, что для более старых машин величина
пробега меньше интересует покупателей, чем для более новых. Какая из приведенных
ниже моделей позволит ему проверить свою гипотезу и каким образом? Можно ли считать
полученный результат доказательством гипотезы посредника? Посредник верит, что
выполняются все основные гипотезы модели линейной регрессии, в том числе гипотеза о
нормальном распределении случайной составляющей.
\begin{enumerate}
\item $Price_t=c_0+c_1 Run_t+c_2 Age_t+w_t$
\item $Price_t=c_0+c_1 Run_t+c_2 Age_t+c_3 Run_t Age_t+w_t$
\item $Price_t=c_0+c_1 Run_t+c_2 Age_t^2+w_t$
\item $Price_t=c_0+c_1 Run_t+c_2 \ln(Run_t Age_t)+w_t$
\end{enumerate}

\item[Проблема 2.] Посреднику необходимо оценить среднестатистический автомобиль, пробег
которого составляет 49,52 тыс. км. Такого автомобиля в его базе еще нет. Если он укажет
неверную <<вилку цен>>, то сделка не состоится. Посредник может позволить себе
ошибиться в среднем в пяти случаях из ста. Какие границы цен он должен назначить, если
для грубой оценки стоимости автомобиля посредник использует модель
$\widehat{Price}_t=\underset{(0.412)}{1.304}+\underset{(0.007)}{0.054}Run_t$? Подойдет ли эта оценка для автомобиля с пробегом 80 тыс. км?


Здесь в скобках стоят стандартные ошибки оценок. Оценка стандартной ошибки
случайной составляющей $s=\sqrt{s^2}\approx 1,660$. Ковариационная матрица оценок
коэффициентов имеет вид 
\[
C=\left(
\begin{array}{cc}
0.17 & 0 -0.003 \\
-0.003 & 0.00005
\end{array}
\right)
\]
Посредник верит, что случайная составляющая имеет нормальное распределение.
\end{enumerate}

\end{enumerate}


\subsection{Олимпиада март-апрель 2014}

\newcommand{\solution}{\subsubsection*{Решение}}

\begin{center}
\textbf{Задание состоит из 10 задач. Время выполнения --- 180 минут.}
\end{center}

\vspace{40pt}




\begin{enumerate}



\item Найдите предел

\[
\lim_{x\to 2014-0} (2014-x)^{\cos \frac{\pi (2015-x)}{2} } 
\]



\solution

За полностью правильно решенную задачу ставилось 10 баллов.

За правильное начало решения в случае "запутывания" при применении правила Лопиталя ставилось 5-6 баллов.

За каждую грубую ошибку снимался 1 балл.

Если делалась осмысленная замена, то ставился 1 балл.

Если утверждалось что 0 в степени 0 равен 1, то ставился 1 балл.

Возможный вариант решения:

$\mathop{\lim }\limits_{x\to 2014-0} (2014-x)^{\cos \frac{\pi (2015-x)}{2} } =\exp \left\{\mathop{\lim }\limits_{x\to 2014-0} \left(\cos \frac{\pi (2015-x)}{2} \cdot \ln (2014-x)\right)\right\}=e^{0} =1,$ т.к.

\[\mathop{\lim }\limits_{x\to 2014-0} \left(\cos \frac{\pi (2015-x)}{2} \cdot \ln (2014-x)\right)=\mathop{\lim }\limits_{x\to 2014-0} \frac{\ln (2014-x)}{\left(\cos \frac{\pi (2015-x)}{2} \right)^{-1} } =\] 

По правилу Лопиталя

\[=\mathop{\lim }\limits_{x\to 2014-0} \frac{\cos ^{2} \frac{\pi (2015-x)}{2} }{(2014-x)\cdot \frac{\pi }{2} \cdot \sin \frac{\pi (2015-x)}{2} } =\frac{2}{\pi } \mathop{\lim }\limits_{x\to 2014-0} \frac{\cos ^{2} \frac{\pi (2015-x)}{2} }{2014-x} =\] 

\[=\mathop{\lim }\limits_{x\to 2014-0} \frac{2\cos \frac{\pi (2015-x)}{2} \sin \frac{\pi (2015-x)}{2} }{-1} =0\] 



Возможный вариант решения:

Обозначим $2014-x=y$. Тогда 

\[\begin{array}{l} {\mathop{\lim }\limits_{x\to 2014-0} (2014-x)^{\cos \frac{\pi (2015-x)}{2} } =\mathop{\lim }\limits_{y\to 0+} y^{\cos ^{\frac{\pi (1+y)}{2} } } =\mathop{\lim }\limits_{y\to 0+} y^{-\sin \frac{\pi y}{2} } =} \\ {=\exp \left\{\mathop{\lim }\limits_{y\to 0+} \left(-\sin \frac{\pi y}{2} \cdot \ln y\right)\right\}=e^{0} =1,} \end{array}\] 

Т.к. $\begin{array}{l} {\mathop{\lim }\limits_{y\to 0+} \left(\sin \frac{\pi y}{2} \cdot \ln y\right)=\mathop{\lim }\limits_{y\to 0+} \frac{\sin \frac{\pi y}{2} }{\frac{\pi y}{2} } \cdot \mathop{\lim }\limits_{y\to 0+} \left(\frac{\pi y}{2} \cdot \ln y\right)=\frac{\pi }{2} \mathop{\lim }\limits_{y\to 0+} \left(y\cdot \ln y\right)=} \\ {=\frac{\pi }{2} \mathop{\lim }\limits_{y\to 0+} \frac{\ln y}{y^{-1} } =\frac{\pi }{2} \mathop{\lim }\limits_{y\to 0+} \frac{y^{2} }{y} =0} \end{array}$


\vspace{6pt}

\item Найдите собственные числа и собственные векторы оператора $(x-2014)^{2} \frac{d^{2} }{dx^{2} } $, действующего в пространстве многочленов степени не выше 4.



\solution

За полностью правильно решенную задачу ставилось 10 баллов.

За правильное нахождение всех собственных чисел ставилось 6 баллов.

Если была правильно найдена часть собственных векторов и все собственные числа, то ставилось 7-8 баллов.

Если была правильно выписана матрица оператора (и указывался базис, в котором матрица имеет соответствующий вид), то ставилось 2 балла.

Если было приведено правильное определение собственных чисел и собственных векторов оператора, то ставился 1 балл.

 

\begin{enumerate}
\item Заметим, что 

\[(x-2014)^{2} \frac{d^{2} }{dx^{2} } (x-2014)^{k} =k(k-1)(x-2014)^{k} ,\; k=0,1,2...\] 

\item  Выберем в качестве базиса в пространстве многочленов степени не выше 4 следующие многочлены: $1,\; (x-2014),\; (x-2014)^{2} ,\; (x-2014)^{3} ,\; (x-2014)^{4} $.

\item  Матрица оператора $(x-2014)^{2} \frac{d^{2} }{dx^{2} } $ в этом базисе имеет вид:

\[\left(\begin{array}{l} {0\; 0\; 0\; 0\; 0} \\ {0\; 0\; 0\; 0\; 0} \\ {0\; 0\; 2\; 0\; 0} \\ {0\; 0\; 0\; 6\; 0} \\ {0\; 0\; 0\; 0\; 12} \end{array}\right)\] 

\item  Т.к. матрица является диагональной, то ее собственные числа -- это диагональные элементы: 0 (кратности 2), 2, 6, 12 (кратности 1), а соответствующие собственные векторы:
\end{enumerate}

Для 0 - $1,\; (x-2014),\; $

Для 2 - $\; (x-2014)^{2} $,

Для 6 - $(x-2014)^{3} $,

Для 12 - $(x-2014)^{4} $.



\vspace{6pt}

\item  Исследуйте на экстремумы функцию

\[
F(x,y)=x^{2} y+3y^{2} x+3y^{3} -8x-25y
\]


\solution


Найдем точки, подозрительные на экстремум, решая следующую систему уравнений $[$1 балл]:

\[\left\{\begin{array}{c} {\frac{\partial F}{\partial x} =2xy+3y^{2} -8=0} \\ {\frac{\partial F}{\partial y} =x^{2} +6xy+9y^2-25=0} \end{array}\right. .\] 

Получим точки со следующими координатами: $\left(1;4/3\right),\left(1;-2\right),\left(-1;2\right),\left(-1;-4/3\right)$ $[$4 балла].

Далее необходимо проверить выполнение условий второго порядка. Для этого найдем матрицу вторых производных исследуемой функции $[$1 балл]:

\[\left(\begin{array}{cc} {2y} & {2x+6y} \\ {2x+6y} & {6x+18y} \end{array}\right).\] 

Проверим знакоопределенность этой матрицы в каждой из найденных подозреваемых точек.

Для точки с координатами $\left(1;4/3\right)$ имеем следующую матрицу: $\left(\begin{array}{cc} {8/3} & {10} \\ {10} & {30} \end{array}\right)$. Так как$8/3\cdot 30-10^{2} <0$, точка с координатами $\left(1;4/3\right)$ не является точкой экстремума $[$1 балл].

Для точки с координатами $\left(1;-2\right)$ имеем следующую матрицу: $\left(\begin{array}{cc} {-4} & {-10} \\ {-10} & {-30} \end{array}\right)$. Так как$-4<0$, $4\cdot 30-10^{2} >0$, точка с координатами $\left(1;-2\right)$ является точкой максимума. Значение функции в этой точке: $F(x,y)=28$ $[$1 балл].

Для точки с координатами $\left(-1;-4/3\right)$ имеем следующую матрицу: $\left(\begin{array}{cc} {-8/3} & {-10} \\ {-10} & {-30} \end{array}\right)$. Так как$8/3\cdot 30-10^{2} <0$, точка с координатами $\left(-1;-4/3\right)$ не является точкой экстремума $[$1 балл].

Для точки с координатами $\left(-1;2\right)$ имеем следующую матрицу: $\left(\begin{array}{cc} {4} & {10} \\ {10} & {30} \end{array}\right)$. Так как$4>0$, $4\cdot 30-10^{2} >0$, точка с координатами $\left(1;-2\right)$ является точкой минимума. Значение функции в этой точке: $F(x,y)=-28$ $[$1 балл].





\vspace{6pt}

\item Для каждого значения параметра $a$ исследуйте функцию 
\[
Q(x,y;a)=\frac{(x+2y)^2-4a(x+a)-x^2}{y-a},
\]
на условные экстремумы при ограничении $F(x,y;a)=x^2+y^2-a^2=0$.


\solution

Выписывание функции Лагранжа - 1 балл.

Нахождение критических точек функции Лагранжа - 5 баллов.

Анализ критических точек - 5 баллов.


\begin{enumerate}
\item  Путем алгебраических преобразований числителя получаем, что $Q\left(x,y;a\right)=4\left(x+y+a\right),y\ne a$

\item  Определяем наличие условных стационарных точек. Для этого выписываем функцию Лагранжа $L\left(x,y,\lambda ;a\right)=\left(x+y+a\right)+\lambda \left(x^{2} +y^{2} -a^{2} \right)$. Приравниваем его производные по $x,y,\lambda $к нулю и получаем два решения: A: $\left(x_{1} ,y_{1} ,\lambda _{1} \right)=\left(\frac{a}{\sqrt{2} } ,\frac{a}{\sqrt{2} } ,-\frac{2\sqrt{2} }{a} \right)$ и B: $\left(x_{2} ,y_{2} ,\lambda _{2} \right)=\left(-\frac{a}{\sqrt{2} } ,-\frac{a}{\sqrt{2} } ,\frac{2\sqrt{2} }{a} \right)$. Очевидно, что выполняется $y\ne a$.

\item  Определяем тип найденных условных стационарных точек. Для этого выписываем второй дифференциал функции Лагранжа $d^{2} L\left(x,y,\lambda ;a\right)=\frac{\partial ^{2} }{\partial x^{2} } L\left(x,y,\lambda ;a\right)dx^{2} +2\frac{\partial ^{2} }{\partial x\partial y} L\left(x,y,\lambda ;a\right)dxdy+\frac{\partial ^{2} }{\partial y^{2} } L\left(x,y,\lambda ;a\right)dy^{2} $. Из ограничения следует, что $dy=-\left({\frac{\partial }{\partial x} F\left(x,y;a\right) \mathord{\left/{\vphantom{\frac{\partial }{\partial x} F\left(x,y;a\right) \frac{\partial }{\partial y} F\left(x,y;a\right)}}\right.\kern-\nulldelimiterspace} \frac{\partial }{\partial y} F\left(x,y;a\right)} \right)dx$. Таким образом, тип стационарной точки определяется знаком выражения 

$\begin{array}{l} {A=\frac{\partial ^{2} }{\partial x^{2} } L\left(x_{k} ,y_{k} ,\lambda _{k} ;a\right)+2\frac{\partial ^{2} }{\partial x\partial y} L\left(x_{k} ,y_{k} ,\lambda _{k} ;a\right)\left({\frac{\partial }{\partial x} F\left(x_{k} ,y_{k} ;a\right) \mathord{\left/{\vphantom{\frac{\partial }{\partial x} F\left(x_{k} ,y_{k} ;a\right) \frac{\partial }{\partial y} F\left(x_{k} ,y_{k} ;a\right)}}\right.\kern-\nulldelimiterspace} \frac{\partial }{\partial y} F\left(x_{k} ,y_{k} ;a\right)} \right)+} \\ {+\frac{\partial ^{2} }{\partial y^{2} } L\left(x_{k} ,y_{k} ,\lambda _{k} ;a\right)\left({\frac{\partial }{\partial x} F\left(x_{k} ,y_{k} ;a\right) \mathord{\left/{\vphantom{\frac{\partial }{\partial x} F\left(x_{k} ,y_{k} ;a\right) \frac{\partial }{\partial y} F\left(x_{k} ,y_{k} ;a\right)}}\right.\kern-\nulldelimiterspace} \frac{\partial }{\partial y} F\left(x_{k} ,y_{k} ;a\right)} \right)^{2} } \end{array}$. 


В данном случае $A=2\lambda _{k} +2\lambda _{k} \left(\frac{x}{y} \right)^{2} =2\lambda _{k} \left(1+\left(\frac{x}{y} \right)^{2} \right)$. Таким образом, тип стационарной точки определяется знаком $\lambda _{k} $. Знак определяется знаком параметра «a». Если $a>0$, то первая стационарная точка A -- точка максимума, вторая точка B -- минимума. Если $a<0$, то первая стационарная точка -- точка минимума, вторая -- максимума. Однако, как не трудно видеть, решения в первом и втором случае совпадают.

%\item  Ответ. Заданная в условии функция при определенном в условии ограничении имеет две точки стационарности $\left(x_{1} ,y_{1} ,\lambda _{1} \right)=\left(\frac{a}{\sqrt{2} } ,\frac{a}{\sqrt{2} } ,\frac{2\sqrt{2} }{a} \right),\; \left(x_{2} ,y_{2} ,\lambda _{2} \right)=\left(-\frac{a}{\sqrt{2} } ,-\frac{a}{\sqrt{2} } ,-\frac{2\sqrt{2} }{a} \right)$. Если $a>0$, то первая стационарная точка -- точка минимума, вторая -- максимума. Если $a<0$, то первая стационарная точка -- точка максимума, вторая -- минимума. Однако, как не трудно видеть, решения в первом и втором случае совпадают.
\end{enumerate}





\vspace{6pt}

\item Задано дифференциальное уравнение
 \[
 y'''+5y''+11y'+15y=15x+56
 \]
\begin{enumerate}
\item Найдите общее решение дифференциального уравнения
\item Найдите асимптоту при $x\to +\infty$ тех решений, у которых асимптота существует
\end{enumerate} 

\solution


\begin{enumerate}
\item $[$1 балл] Составляем характеристическое уравнение
\[
\lambda^3+5\lambda^2+11\lambda+15=0
\]
\item $[$3 балла]  Перебором делителей свободного члена находим отрицательный корень $\lambda=-3$. После этого находим все корни: $\lambda_1=-3$, $\lambda_2=-1+2i$, $\lambda_3=-1-2i$.
$[$1 балл]  Решение однородного:
\[
y_{hom}(x)=c_1e^{-3x}+e^{-x}(c_2\cos(2x)+c_3\sin(3x))
\]
\item $[$3 балла]  По виду правой части находим частное решение, $y_{ps}(x)=x+3$
\item Выписываем общее решение
\[
y(x)=c_1e^{-3x}+e^{-x}(c_2\cos(2x)+c_3\sin(3x))+x+3
\]
\item $[$2 балла]  Асимптота равна $y_{as}(x)=x+3$
\end{enumerate}


\item Васе нравятся Маша и Алёна, поэтому он ходит на лекции, чтобы пообщаться с ними. Вероятность застать Васю на лекциях равна $0.1$, если девушек нет; $0.5$ --- если обе девушки пришли на лекции; $0.3$ --- если пришла только Маша и $0.2$ --- если пришла только Алёна. Маша и Алёна посещают лекции независимо друг от друга с вероятностями $0.6$ и $0.7$ соответственно.


\begin{enumerate}
\item $[$2 балла]  Найдите вероятность того, что на лекциях будет присутствовать ровно одна из девушек
\item $[$2 балла]  Найдите вероятность того, что на лекциях присутствуют все трое
\item $[$3 балла]  Найдите вероятность того, что на лекциях присутствует Алёна, если на лекциях присутствует Вася
\item $[$3 балла] Найдите вероятность того, что на лекциях присутствует Вася, если пришла ровно одна из девушек
\end{enumerate}

\solution


Обозначим: величина $X$ --- количество пришедших девушек, событие $V$ --- пришел Вася, $A$ --- пришла Алёна и $M$ --- Маша. 
\begin{enumerate}
\item $\P(X=1)=0.6\cdot 0.3+0.4\cdot 0.7=0.46$
\item $\P(A\cap M\cap V)=0.6\cdot 0.7\cdot 0.5=0.21$
\item По формуле условной вероятности:
\[
\P(A|V)=\P(A\cap V)/\P(V)=\frac{6\cdot 7\cdot 5 + 4\cdot 7 \cdot 2}
{6\cdot 7\cdot 5 + 4\cdot 7 \cdot 2+ 6\cdot 3\cdot 3 + 4\cdot 3\cdot 1}=0.801
\]
\item По формуле условной вероятности:
\[
\P(V|X=1)=\P(V\cap \{X=1\})/\P(X=1)=\frac{0.6\cdot 0.3\cdot 0.3+0.4\cdot 0.7\cdot 0.2}{0.46}=0.239
\]
\end{enumerate}

Если условные вероятности посчитаны неверно, но выписана правильная формула для условной вероятности, то начислялся один балл.

\vspace{6pt}

\item Совместная функция плотности случайных величин $X$ и $Y$ имеет вид:

\[
f(x,y)=
\begin{cases}
x+\frac{3}{2} y^{2}, \, \text{если } x\in [0;1],\, y\in[0;1] \\
0, \text{иначе} 
\end{cases}
\] 

\begin{enumerate}
\item  Найдите $\E(X)$, $\E(X^{2})$, дисперсию $\Var(X)$

\item Найдите функцию плотности величины $X$

\item Найдите вероятности $\P(Y<\frac{1}{2} \mid X<\frac{1}{2} )$ и $\P(Y>2X)$.

\item Найдите константу $c$, если $g(x,y)=cxf(x,y)$ --- это совместная функция плотности для некоторой пары случайных величин
\end{enumerate}



\solution


(a) и (b) Частную функцию плотности найдем по формуле $f_{X} (x)=\int _{0}^{1}x+\frac{3}{2} y^{2}  dy=x+\frac{1}{2} ,x\in \left[0,1\right]$ $[$2 балла].

Математические ожидания найдем интегрированием $\E(X)=\int _{0}^{1}x\left(x+\frac{1}{2} \right) dx=\frac{7}{12} $ $[$1 балл],  $\E(X^{2} )=\int _{0}^{1}x^{2} \left(x+\frac{1}{2} \right) dx=\frac{5}{12} $ $[$1 балл]. Тогда дисперсию можно рассчитать как разность  $\Var(X)=\E(X^{2} )-\E^{2} (X)=\frac{11}{144} $ $[$1 балл].


(c) Условную вероятность рассчитаем по формуле как отношение $[$2 балла]

\[\P\left(Y<\frac{1}{2} \left|X<\frac{1}{2} \right. \right)=\frac{\P\left(Y<\frac{1}{2} ,X<\frac{1}{2} \right)}{\P\left(X<\frac{1}{2} \right)} =\frac{\int _{0}^{0,5}\int _{0}^{0,5}x+\frac{3}{2} y^{2}   dxdy}{\int _{0}^{0,5}x+\frac{1}{2}  dx} =\frac{\frac{3}{32} }{\frac{3}{8} } =\frac{1}{4} .\] 

Представим требуемую вероятность как интеграл со следующими пределами интегрирования $[$2 балла]

\[\begin{array}{l} {\P\left(Y>2X\right)=\int _{0}^{0,5}\int _{2x}^{1}x+\frac{3}{2} y^{2}   dydx=\int _{0}^{0,5}\left. xy+\frac{1}{2} y^{3} \right|_{2x}^{1} dx =} \\ {\int _{0}^{0,5}x+\frac{1}{2} -2x^{2} -4x^{3} dx= \frac{1}{8} +\frac{1}{4} -\frac{1}{12}  -\frac{1}{16} =\frac{11}{48} } \end{array}.\] 

(d) Неизвестную константу найдем из условия  $c\int _{0}^{1}\int _{0}^{1}x^2+\frac{3}{2} xy^{2}   dxdy=1$. Получаем $c=12/7$ $[$1 балл]




\vspace{6pt}

\item  В случайной выборке $X_1$, $X_2$, \ldots, $X_n$ величины $X_i$ независимы и принимают значения 0 и 1, вероятность события $\{X_i=1\}$ обозначим $p$. Для проверки гипотезы $H_0$: $p=1$  против альтернативы $H_a$: $p<1$ используется критерий: отвергать $H_0$, если $\sum_{i=1}^n X_i < n$.
\begin{enumerate}
\item Предположим, что на самом деле $p=1$. С какой вероятностью основная гипотеза будет отвергнута?
\item Пусть $0<p<1$. Выпишите мощность критерия как функцию от $n$ и $p$.
\end{enumerate}

\solution


Обозначим $\sum_{i=1}^n X_i < n$ как событие $A$. Его можно выразить словами так: <<хотя бы одна из величин $X_i$ принимает значение 0>>. 

а) При $p=1$ вероятность отвергнуть основную гипотезу будет равна нулю. Событие $A$ невозможно --- все $X_i$ достоверно равны единице, так что $\sum_{i=1}^n X_i = n$.

\textit{Пункт (а) оценивается в 5 баллов.}

б) Мощность критерия – вероятность отвергнуть основную гипотезу, когда верна альтернативная. В случае $0 \leq p <1$ основная гипотеза отвергается с вероятностью $\P(A)=1-\P(\bar{A})=1-p^n$  --- это и есть функция мощности.

\textit{Пункт (б) оценивается в 5 баллов. За понимание того, что такое мощность, ставится 2  балла.}

\item Известно, что объём порции кофе, приготовленной кофейным аппаратом, имеет нормальное распределение со стандартным отклонением 2 мл и математическим ожиданием, которое устанавливает производитель аппарата. Любительница кофе Соня не понимает, какой средний объём порции установлен на её аппарате, и решает приготовить 9 порций кофе, средний объём которых оказывается равен $\bar{X}=51.8$ мл.
\begin{enumerate}
\item Рассчитайте 90\% доверительный интервал для математического ожидания объёма.
\item  Предположим, что Соня решает пользоваться доверительным интервалом вида $(\bar{X}-2;\bar{X}+2)$. 

Какой доверительной вероятности он соответствует?
\end{enumerate}

\solution


Пользуемся доверительным интервал для математического ожидания в случае нормальной генеральной совокупности с известной дисперсией:

\[
\bar{X}-z_{\alpha/2}\frac{\sigma}{\sqrt{n}} <\mu<\bar{X}+z_{\alpha/2}\frac{\sigma}{\sqrt{n}}
\]
Табличное значение $z_{0.1/2}=1.645$ (по таблицам. нормального распределения можно определить, что $1.64<z_{0.1/2}<1.65$). Подставляя данные из условия, получаем интервал:

\[
51.8-1.645\frac{2}{\sqrt{9}}<\mu <51.8+1.645\frac{2}{\sqrt{9}}
\]

\[
50.703 < \mu < 52.897
\]


\textit{Пункт (а) оценивается в 5 баллов, за ошибку при использовании таблиц снимается 2 балла.}

Интервал $(\bar{X}-2;\bar{X}+2)$ --- частный случай интервала из части (а), где $z_{\alpha/2}\frac{\sigma}{\sqrt{n}}=2$. Поэтому $z_{\alpha/2}=\frac{2\sqrt{n}}{\sigma}=\frac{2\sqrt{9}}{2}=3$. Посмотрев в таблицы нормального распределения, видим, что это значение соответствует доверительной вероятности в 99.73\% (правило трёх сигм).

\textit{Пункт (б) оценивается в 5 баллов, за ошибку при использовании таблиц снимается 2 балла.}


\vspace{6pt}

\item Менеджер Петров планирует купить подержанный автомобиль и накопил для этого некоторую сумму --- 1000 у.е. Он выбрал тип автомобиля: производителя, модель и комплектацию. В качестве основного параметра для принятия решения он рассматривает пробег (переменная <<Run>>) в тысячах километров. Петров собрал данные о ценах на 100 автомобилей выбранного типа в единицах у.е. с различным пробегом и решил для анализа своих возможностей воспользоваться линейной регрессией $Price_t=\beta_1+\beta_2 Run_t+v_t$. Предположим, что все предпосылки классической нормальной регрессионной модели выполнены. Результаты оценивания модели по имеющимся у Петрова данным: 
\begin{itemize}
\item оценки значений параметров, $\hat{\beta}_1=2000$, $\hat{\beta}_2=-2$
\item оценка ковариационной матрицы оценок значений параметров, $\hat{C}=
\begin{pmatrix}
2 & 1 \\
1 & 2 
\end{pmatrix}$
%\todo[inline]{Ахтунг! Данные противоречивы! В данном случае при обращении $\hat{C}/\hat{\sigma}^2$ получается дробное количество наблюдений меньше 1}
\item оценка дисперсии случайной составляющей $\hat{\sigma}^2=150$.
\end{itemize}

%Все оценки значимы на 95\% уровне.
\begin{enumerate}
\item С вероятностью ошибки первого рода 5\% проверьте гипотезу менеджера о том, что за каждую тысячу километров пробега автомобиль теряет в цене не две, а две с половиной у.е. 
\item Каков минимальный пробег автомобиля выбранного типа, начиная с которого Петрову хватит денег на покупку с вероятностью  97.5\%?
\end{enumerate}
%\todo[inline]{Добавить простые вопросы, типа проверьте гипотезу, что бета2=7}


\solution

Формализация (выписывание в виде математических соотношений) любой из проблем Петрова - 2 балла. Решение любой из проблем - 5 баллов

Конкретная величина баллов может быть иной в зависимости от полноты решения и допущенных ошибок.


\underbar{Теория}. Решение основано на построении доверительного интервала для нового значения зависимой переменной -- цены (\textit{Price}). С заданной доверительной вероятностью он накрывает искомое новое значение. Поскольку Петров согласен с вероятностью покупки 0.97725, то доверительная вероятность равна этой величине. Необходимо построить указанный выше доверительный интервал при произвольном значении независимой переменной -- пробеге (\textit{Run}). Далее, приравнивая нижний предел данного интервала к имеющейся у Петрова сумме, определяем минимальный пробег автомобиля, на который он может рассчитывать.  

\underbar{Расчеты}. 

\begin{enumerate}
\item  Для модели линейной регрессии приближенный (на основе оценок) доверительный интервал для нового значения зависимой переменной имеет вид: $P\left(Price\in \left[\left(\hat{a},x\right)-u_{\alpha } \sqrt{x'\hat{C}x+s^{2} } ,\left(\hat{a},x\right)+u_{\alpha } \sqrt{x'\hat{C}x+s^{2} } \right]\right)\approx \alpha $. Подставляя в это выражение имеющиеся у Петрова результаты оценивания, и учитывая указанное значение квантили, получаем нижнюю границу доверительного интервала:

\[\begin{array}{l} {Price_{L} \left(Run\right)=2000.0-2.0\cdot Run-2\cdot \sqrt{\left(2.0+2.0\cdot Run+2.0\cdot Run^{2} \right)+150} =} \\ {=2000.0-2.0\cdot Run-2\cdot \sqrt{152+2.0\cdot Run+2.0\cdot Run^{2} } } \end{array}\] 

\item  Приравнивая найденное выражение накопленной сумме, получаем уравнение:
$500- Run= \sqrt{152+2.0\cdot Run+2.0\cdot Run^{2} } $.  Выражение под знаком квадратного корня в правой части равенства всегда положительно (корней нет). Возводим левую и правую часть в квадрат. В результате получаем уравнение $Run^{2} +1002\cdot Run-249848=0$. Оно имеет два корня $Run_1= -683.71$, $Run_2=182.71$. Очевидно, что второй корень равен искомой величине.

\end{enumerate}

Решение второй проблемы Петрова.


Теория. Решение основано на проверке гипотезы о величине математического ожидания нормального распределения при известной дисперсии. (Возможен вариант решения при неизвестной дисперсии, но для этого потребуются таблицы или их эквивалент для распределения Стьюдента. Если участник Олимпиады пойдет этим путем, можно добавить несколько баллов).  


Расчеты.  Проверяется гипотеза $H_0$: $\beta_2=-2.5$ против $H_a$: $\beta_2 \neq -2.5$. При сделанных предположениях и полученных оценках критическая область имеет вид $W=(-\infty,z_L]\cup [z_U;+\infty)$, где $z_L=-2.5-1.96\cdot 1.41\approx -5.26$, $z_U=-2.5+1.96\cdot 1.41\approx 0.26$.  Полученная оценка значения параметра  $\hat{\beta}_2$ не попадает в критическую область. Таким образом, гипотеза менеджера Иванова не отвергается.


\end{enumerate}

\vspace{30pt}

\textbf{Справочная информация:} 

\vspace{6pt}

Если $F()$ --- функция распределения стандартной нормальной случайной величины, то $F(1)\approx 0.841$, $F(1.282)\approx 0.9$, $F(1.645)\approx 0.95$, $F(1.96)\approx 0.975$, $F(2.241)\approx 0.9875$, $F(3)\approx 0.9987$.

При необходимости можно использовать линейную интерполяцию для нахождения нужных квантилей.





\end{document}
